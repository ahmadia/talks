% \documentclass[handout]{beamer}
\documentclass{beamer}

\mode<presentation>
{
  \usetheme{default}
  \usefonttheme[onlymath]{serif}
  % \usetheme{Singapore}
  % \usetheme{Warsaw}
  % \usetheme{Malmoe}
  % \useinnertheme{circles}
  % \useoutertheme{infolines}
  % \useinnertheme{rounded}

  \setbeamercovered{transparent=100}
}

\usepackage[english]{babel}
\usepackage[latin1]{inputenc}
\usepackage{textpos,alltt,listings,multirow,ulem,siunitx}
\usepackage{pdfpages}
\usepackage{multimedia}
\newcommand\hmmax{0}
\newcommand\bmmax{0}
\usepackage{bm}

% font definitions, try \usepackage{ae} instead of the following
% three lines if you don't like this look
\usepackage{mathptmx}
\usepackage[scaled=.90]{helvet}
% \usepackage{courier}
\usepackage[T1]{fontenc}
\usepackage{tikz}
\usetikzlibrary[shapes,shapes.arrows,arrows,shapes.misc,fit,positioning,trees,mindmap,backgrounds]

% \usepackage{pgfpages}
% \pgfpagesuselayout{4 on 1}[a4paper,landscape,border shrink=5mm]

\usepackage{JedMacros}
\usepackage{ctmmath-v3}

\title{Preconditioner Scaling for Finite Element Models of Turbulent Air/Water Flow in Coastal and Hydraulic Applications}
\author{{\bf Christopher E. Kees}\inst{1},\\
Aron J. Ahmadia\inst{1}, Jed Brown\inst{2}, Barry Smith\inst{2}}


% - Use the \inst command only if there are several affiliations.
% - Keep it simple, no one is interested in your street address.
\institute
{
  \inst{2}{Coastal and Hydraulics Laboratory, US Army Engineer
    Research and Development Center} \\
  \inst{3}{Mathematics and Computer Science Division, Argonne National Laboratory} \\
}

\date{SIAM CSE 2015-03-18}

% This is only inserted into the PDF information catalog. Can be left
% out.
\subject{Talks}


% If you have a file called "university-logo-filename.xxx", where xxx
% is a graphic format that can be processed by latex or pdflatex,
% resp., then you can add a logo as follows:

% \pgfdeclareimage[height=0.5cm]{university-logo}{university-logo-filename}
% \logo{\pgfuseimage{university-logo}}



% Delete this, if you do not want the table of contents to pop up at
% the beginning of each subsection:
% \AtBeginSubsection[]
% {
% \begin{frame}<beamer>
%   \frametitle{Outline}
%   \tableofcontents[currentsection,currentsubsection]
% \end{frame}
% }

\AtBeginSection[]
{
  \begin{frame}<beamer>
    \frametitle{Outline}
    \tableofcontents[currentsection]
  \end{frame}
}

% If you wish to uncover everything in a step-wise fashion, uncomment
% the following command:

% \beamerdefaultoverlayspecification{<+->}

\begin{document}
\lstset{language=C}
\normalem

\begin{frame}
  \titlepage
\end{frame}

\begin{frame}{Definitions}
\begin{description}
\item [$\rho$] density
\item[$t$] time
\item[$\vec u$] fluid velocity
\item[$p$] fluid pressure
\item[$\mu$] dynamic viscosity
\item[$\vec g$] gravitational acceleration vector.
\end{description}
\end{frame}
\begin{frame}{Governing Equations}
  \begin{block}{Navier Stokes - immiscible, incompressible fluids}
    \vspace{-1em}
    \begin{align*}
      \pdt{\vec u}
      +\del \cdot \left[ \vec u \otimes \vec u -
      \nu(\theta) \del^s\vec u \right]+ \frac{\del p}{\rho(\theta)}  = \vec f \\
      \deld \vec u = 0
    \end{align*}
  \end{block}
  \begin{block}{Volume Fraction}
    \begin{align*}
      \theta_t + \nabla \cdot (\theta  u) = 0
    \end{align*}
  \end{block}
  \begin{block}{Constitutive Relationship}
    \begin{align*}
      \rho = \rho_w\theta + \rho_a[1 - \theta] \\
      \nu = \nu_w\theta + \nu_a[1 - \theta]
    \end{align*}
  \end{block}
      \vspace{-0.8em}
  {\scriptsize (Kees, Akkerman, Farthing, Bazilevs. \emph{A
      conservative level set method suitable for variable-order
      approximations and unstructured meshes}. 2011)}
\end{frame}

\input{slides/TwoPhaseNavierStokesLES/Coupling.tex}
\input{slides/FieldSplit.tex}
\input{slides/Stokes/OptionsTour.tex}

\begin{frame}{Conclusion}
  Notes here...
\end{frame}
\end{document}
