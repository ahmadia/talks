% \documentclass[handout]{beamer}
\documentclass{beamer}

\mode<presentation>
{
  \usetheme{ANLBlue}
  % \usefonttheme[onlymath]{serif}
  % \usetheme{Singapore}
  % \usetheme{Warsaw}
  % \usetheme{Malmoe}
  % \useinnertheme{circles}
  % \useoutertheme{infolines}
  % \useinnertheme{rounded}

  \setbeamercovered{transparent=20}
}

\usepackage[english]{babel}
\usepackage[latin1]{inputenc}
\usepackage{alltt,listings,multirow,ulem,siunitx}
\usepackage[absolute,overlay]{textpos}
\TPGrid{1}{1}
\usepackage{pdfpages}
\usepackage{ulem}
\usepackage{multimedia}
\usepackage{multicol}
\newcommand\hmmax{0}
\newcommand\bmmax{0}
\usepackage{bm}
\usepackage{comment}
\usepackage{subcaption}

% font definitions, try \usepackage{ae} instead of the following
% three lines if you don't like this look
\usepackage{mathptmx}
\usepackage[scaled=.90]{helvet}
% \usepackage{courier}
\usepackage[T1]{fontenc}
\usepackage{tikz}
\usetikzlibrary{decorations.pathreplacing}
\usetikzlibrary{shadows,arrows,shapes.misc,shapes.arrows,shapes.multipart,arrows,decorations.pathmorphing,backgrounds,positioning,fit,petri,calc,shadows,chains,matrix}

\newcommand\vvec{\bm v}
\newcommand\bvec{\bm b}
\newcommand\bxk{\bvec_0 \times \kappa_0 \cdot \nabla}
\newcommand\delp{\nabla_\perp}

% \usepackage{pgfpages}
% \pgfpagesuselayout{4 on 1}[a4paper,landscape,border shrink=5mm]

\usepackage{JedMacros}

\newcommand{\timeR}{t_{\mathrm{R}}}
\newcommand{\timeW}{t_{\mathrm{W}}}
\newcommand{\mglevel}{\ensuremath{\ell}}
\newcommand{\mglevelcp}{\ensuremath{\mglevel_{\mathrm{cp}}}}
\newcommand{\mglevelcoarse}{\ensuremath{\mglevel_{\mathrm{coarse}}}}
\newcommand{\mglevelfine}{\ensuremath{\mglevel_{\mathrm{fine}}}}

%solution and residual
\newcommand{\vx}{\ensuremath{x}}
\newcommand{\vc}{\ensuremath{\hat{x}}}
\newcommand{\vr}{\ensuremath{r}}
\newcommand{\vb}{\ensuremath{b}}

%operators
\newcommand{\vA}{\ensuremath{A}}
\newcommand{\vP}{\ensuremath{I_H^h}}
\newcommand{\vS}{\ensuremath{S}}
\newcommand{\vR}{\ensuremath{I_h^H}}
\newcommand{\vI}{\ensuremath{\hat I_h^H}}
\newcommand{\vV}{\ensuremath{\mathbf{V}}}
\newcommand{\vF}{\ensuremath{F}}
\newcommand{\vtau}{\ensuremath{\mathbf{\tau}}}


\title{Efficient Implicitness}
\subtitle{Latency-Throughput and Cache-Vectorization Tradeoffs}
\author{{\bf Jed Brown} \texttt{jedbrown@mcs.anl.gov} (ANL and CU Boulder)
}

% - Use the \inst command only if there are several affiliations.
% - Keep it simple, no one is interested in your street address.
% \institute
% {
%   Mathematics and Computer Science Division \\ Argonne National Laboratory
% }

\date{Heterogeneous Multi-Core workshop, NCAR, 2014-09-17 \\[1em]
{\small This talk: \url{http://59A2.org/files/20140917-EfficientImplicitness.pdf}}}

% This is only inserted into the PDF information catalog. Can be left
% out.
\subject{Talks}


% If you have a file called "university-logo-filename.xxx", where xxx
% is a graphic format that can be processed by latex or pdflatex,
% resp., then you can add a logo as follows:

% \pgfdeclareimage[height=0.5cm]{university-logo}{university-logo-filename}
% \logo{\pgfuseimage{university-logo}}



% Delete this, if you do not want the table of contents to pop up at
% the beginning of each subsection:
% \AtBeginSubsection[]
% {
% \begin{frame}<beamer>
%   \frametitle{Outline}
%   \tableofcontents[currentsection,currentsubsection]
% \end{frame}
% }

% \AtBeginSection[]
% {
%   \begin{frame}<beamer>
%     \frametitle{Outline}
%     \tableofcontents[currentsection]
%   \end{frame}
% }

% If you wish to uncover everything in a step-wise fashion, uncomment
% the following command:

% \beamerdefaultoverlayspecification{<+->}

\begin{document}
\lstset{language=C}
\normalem

\begin{frame}
  \titlepage
\end{frame}

\begin{frame}{Intro}
  \begin{itemize}
  \item I work on PETSc, a popular linear and nonlinear solvers library
  \item Some users need fastest time to solution at strong-scaling limit
  \item Others fill memory with a problem for PETSc
  \item Sparse matrices are a dead end for memory bandwidth reasons
    \begin{itemize}
    \item but heavily embraced by legacy code and enable algebraic multigrid
    \end{itemize}
  \item We need to restructure algorithms, but how?
  \item What are the fundamental long-term bottlenecks?
  \item<2> Worrisome trends
    \begin{enumerate}
    \item Fine-grained parallelism without commensurate increase in caches
    \item Emphasizing vectorization over cache reuse
    \item High instruction latency to be covered by hardware threads
    \end{enumerate}
  \end{itemize}
\end{frame}

\begin{frame}{Hardware Arithmetic Intensity}
  \begin{tabular}{lc}
    \toprule
    Operation                         & Arithmetic Intensity (flops/B) \\
    \midrule
    Sparse matrix-vector product      & 1/6                  \\
    Dense matrix-vector product       & 1/4                  \\
    Unassembled matrix-vector product, residual & $\gtrsim 8$          \\
    %High-order residual evaluation    & $\gtrsim 5$                \\
    \bottomrule
  \end{tabular}

  \bigskip

  \begin{tabular}{lrrr}
    \toprule
    Processor & \textsc{stream} Triad (GB/s) & Peak (GF/s) & Balance (F/B) \\
    \midrule
    E5-2680 8-core      & 38   & 173  & 4.5 \\ % http://www.cs.virginia.edu/stream/stream_mail/2013/0001.html
    E5-2695v2 12-core & 45 & 230 & 5.2 \\ % Edison 460 GF and 89 GB/s Triad
    E5-2699v3 18-core & 60 & 660 & 11 \\
    % Magny Cours 16-core & 49   & 281  & 5.7 \\
    Blue Gene/Q node    & 29.3   & 205  & 7 \\ % Morozov's best Triad (https://hpgmg.org/lists/archives/hpgmg-forum/2014-June/000071.html)
    % Tesla M2090         & 120  & 665  & 5.5 \\
    Kepler K20Xm        & 160 & 1310 & 8.2 \\ % http://www.elekslabs.com/2012/11/nvidia-tesla-k20-benchmark-facts.html
    Xeon Phi SE10P      & 161 & 1060 & 6.6 \\ % http://www.cs.virginia.edu/stream/stream_mail/2013/0002.html
    % http://blogs.utexas.edu/jdm4372/tag/xeon-phi/
    \midrule
    KNL (DRAM) & 100 & 3000 & 30 \\
    KNL (MCDRAM) & 500 & 3000 & 6 \\
    \bottomrule
  \end{tabular}
\end{frame}


\begin{frame}{How much parallelism out of how much cache?}
  \begin{tabular}{l rrrr rr}
    \toprule
    Processor & v width & threads & F/inst & latency & L1D & L1D/\#par \\
    \midrule
    Nehalem & 2 & 1 & 2 & 5 & 32 KiB & 1638 B \\
    Sandy Bridge & 4 & 2 & 2 & 5 & 32 KiB & 819 B \\
    Haswell & 4 & 2 & 4 & 5 & 32 KiB & 410 B \\
    BG/P & 2 & 1 & 2 & 6 & 32 KiB & 1365 B \\
    BG/Q & 4 & 4 & 2 & 6 & 32 KiB & 682 B \\
    KNC & 8 & 4 & 4 & 5 & 32 KiB & 205 B \\
    Tesla K20 & 32 & * & 2 & 10 & 64 KiB & 102 B \\
    \bottomrule
  \end{tabular}
  \begin{itemize}
  \item Most ``fast'' algorithms do about $O(n)$ flops on $n$ data
  \item DGEMM and friends do $O(n^{3/2})$ flops on $n$ data
  \item Exploitable parallelism limited by cache and register load/store
  \end{itemize}
\end{frame}

\begin{frame}{Story time: 27pt stencils instruction-limited for BG/P}
  \begin{center}
    \includegraphics[width=0.5\textwidth]{figures/hardware/Malas2012-27pt.png}
  \end{center}
  \begin{itemize}
  \item rolling 2-step kernel extended to 27-point stencil
  \item $2\times 3$ unroll-and-jam used exactly 32 registers
  \item jam width limited by number of registers, barely covers ILP
  \item 200-entry jammed stream fits in L1
    \begin{itemize}
    \item reuse in two directions for most problem sizes
    \end{itemize}
  \item Malas, Ahmadia, Brown, Gunnels, Keyes (IJHPCA 2012)
  \end{itemize}
\end{frame}

\begin{frame}{Fine-grained parallelism in SpMM}
  \begin{figure}
    \centering
    \includegraphics[width=0.8\textwidth]{figures/MG/SACUSPExpand}
  \end{figure}
  \begin{itemize}
  \item Enumerate all scalar products contributing to row of product, $\hat C$
  \item Implemented using \texttt{scan} and \texttt{gather}
  \item Radix sort contributions to each row (two calls to \texttt{sort})
  \item Contract row: \texttt{reduce\_by\_key}
  \item c/o Steve Dalton (2013 Givens Fellow, now at NVidia)
  \end{itemize}
\end{frame}

\begin{frame}{CUSP Performance summary}
  \begin{figure}
    \centering
    \includegraphics[width=0.8\textwidth]{figures/MG/SACUSPSpeedupAP}
  \end{figure}  
  \begin{itemize}
  \item New CUSP SpMM is faster than CUSPARSE for all test matrices.
  \item Sorting optimization faster except for very irregular graph.
  \end{itemize}
\end{frame}

\begin{frame}{Memory overhead from expansion}
  \begin{figure}
    \centering
    \includegraphics[width=0.48\textwidth]{figures/MG/SACUSPExpansionFactor} \quad
    \includegraphics[width=0.48\textwidth]{figures/MG/SACUSPContractionFactor}
    \caption{Scalar Poisson: Expansion factor $nnz(\hat C)/nnz(A)$, contraction $nnz(\hat C)/nnz(C)$}
  \end{figure}
  \vspace{-2em}
  \begin{itemize}
  \item 3D has much higher variability by row
  \item For elasticity, expansion factor is larger by 3x (for 3D)
  \item Implementation could batch to limit total memory usage
    \begin{itemize}
    \item more kernel launches
    \end{itemize}
  \end{itemize}  
\end{frame}

\begin{frame}{Finite element: assembled versus unassembled}
  \vspace{1ex}
  \includegraphics[width=0.9\textwidth]{figures/TensorVsAssembly} \\
  \begin{itemize}
  \item Arithmetic intensity for $\Qk p$ elements
    \begin{itemize}
    \item $< \frac 1 4$ (assembled), $\approx 10$ (unassembled), $\approx 4$ to $8$ (hardware)
    \end{itemize}
  \item store Jacobian information at Quass quadrature points
  \item 70\% of peak for $Q_3$ on Nehalem - vectorization within an element
  \item 30\% of peak for $Q_2$ on Sandy Bridge and Haswell - vectorization across elements
  \end{itemize}
\end{frame}

\begin{frame}{pTatin3d: Lithospheric Dynamics}
  \begin{itemize}
  \item Heterogeneous, visco-plastic Stokes with particles for material composition/chemistry, geometric MG with coarse AMG
  \item May, Brown, Le Pourhiet (SC14)
  \item Viscous operator application for $Q_2$-$P_1^{\text{disc}}$
  \item ``Tensor'': matrix-free implementation using tensor product structure on the reference element
  \item ``Tensor C'' absorbs metric term into stored tensor-valued coefficient
  \item Performance on 8 nodes of Edison (3686 GF/s peak)
  \end{itemize}
  \begin{tabular}{lrrrrrrr}
    \toprule
    Operator & flops & \multicolumn{2}{c}{Pessimal cache} & \multicolumn{2}{c}{Perfect cache} & Time & GF/s \\
    & & bytes & F/B & bytes & F/B & (ms) & \\
    \midrule
    Assembled & 9216 & --- & --- &  37248 & 0.247 & 42 & 113 \\
    Matrix-free & 53622 & 2376 & 22.5 & 1008 & 53 & 22 & 651 \\
    Tensor & 15228 & 2376 & 6.4 & 1008 & 15 & {\bf 4.2} & 1072 \\
    Tensor C & 14214 & 5832 & 2.4 & 4920 & 2.9 & --- & --- \\
    \bottomrule
  \end{tabular}
  % edison-8sinker-64-asm-00192.log:MGResid Level 2      247 1.0 1.0271e+01 1.1 6.50e+09 1.2 8.2e+05 8.3e+03 0.0e+00 10  8  3  7  0  13 14  4 11  0 112942
  % edison-8sinker-64-avx-00192.log:MGResid Level 2      247 1.0 1.0430e+00 1.0 6.01e+09 1.1 1.6e+06 5.5e+03 0.0e+00  3  7  5  9  0   6 14  6 12  0 1072558
  % edison-8sinker-64-ref-00192.log:MGResid Level 2      247 1.0 5.4271e+00 1.0 1.90e+10 1.1 1.6e+06 5.5e+03 0.0e+00  8 12  5  9  0  12 15  6 12  0 651324
\end{frame}

\begin{frame}{Cache versus vectorization}
  \begin{itemize}
  \item Fundamental trade-off
  \item Hardware gives us less cache per vector lane
  \item Intra-element vectorization is complicated and \"uber-custom
  \item Coordinate transformation is $27\cdot 9\cdot \texttt{sizeof(double)} = 1944$ bytes/element.
  \item Vectorize over 4 or 8 elements, perhaps hardware threads
  \item L1 cache is not this big: repeated spills in tensor contraction
  \item This is a \emph{very} simple problem
  \end{itemize}
\end{frame}

% \begin{frame}{Matrix-free operator application}
%   Discretize $-\nabla \Big(\kappa \nabla\cdot u\Big)$, yielding
%   \begin{equation*}\label{eq:mf-scalar}
%     A \mathbf u = \sum_{e \in \text{Elements}} \mathcal E_e^T \mathcal D_{\mathbf x}^T \Lambda(\omega \kappa) \mathcal D_{\mathbf x} \mathcal E_e \mathbf u
%   \end{equation*}
%   Physical gradient matrix
%   $$\mathcal D_{\mathbf x} = \{\mathcal D_i | i \in \{x,y,z\} \} \in \mathbb R^{81\times 27} \quad \text{$Q_2$ elements in 3D; Gauss quadrature}$$
%   \begin{itemize}
%   \item Common method: precompute reference gradient matrix $\mathcal D_{\mathbf \xi}$, map to physical $\mathcal D_{\mathbf x} = \Lambda(\nabla_{\mathbf x}\mathbf\xi) \mathcal D_{\mathbf \xi}$
%   \item Better: rearrange
%   \begin{equation*}\label{eq:tensor}
%     A \mathbf u = \sum_{e \in \text{Elements}} \mathcal E_e^T \mathcal D_{\mathbf \xi}^T \Lambda\Big((\nabla_{\mathbf x}\mathbf\xi)^T (\omega \kappa) (\nabla_{\mathbf x}\mathbf\xi) \Big) \mathcal D_{\mathbf \xi} \mathcal E_e \mathbf u .
%   \end{equation*}
%   where $\nabla_{\mathbf x}\mathbf\xi = (\nabla_{\mathbf\xi}\mathbf x)^{-1}$ is computed at quadrature points from the coordinate gradients.
%   \begin{equation*}
%     D_{\mathbf\xi} = \{\hat D \otimes \hat B \otimes \hat B, \hat B \otimes \hat D \otimes \hat B, \hat B \otimes \hat B \otimes \hat D \}
%   \end{equation*}
%   \end{itemize}
% \end{frame}

% \begin{frame}{Assembly-free preconditioning}
%   \begin{itemize}
%   \item ``Optimization'' is pessimization if it compromises convergence
%   \item pTatin3D: long-term lithosphere/tectonics package
%     \begin{itemize}
%     \item Dave May (ETH Z\"urich) and Laetitia Le Pourhiet (UPMC Paris)
%     \item visco-plastic rheology, post-failure deformation, thermodynamics, free-surface
%     \item multi-material transport using particles; $10^{10}$ jumps in coefficients
%     \end{itemize}
%   \item Block preconditioning with MG solve in viscous block
%   \item Matrix-free fine grid, start coarsening geometrically
%   \item Switch to Galerkin, smoothed aggregation (GAMG or ML)
%   \end{itemize}
%   \begin{tabular}{l rrr rr r}
%     \toprule
%     Operator    & Cores & Grid   & El/core & \multicolumn{2}{c}{Solve/core} & Op/core           \\
%     &       &        &         & El/s                    & GF/s & kEl/s \\
%     \midrule
%     Assembled   & 192 & $64^3$ & 1265 & 46  & 0.9 & 33         \\
%     Matrix-free & 192 & $64^3$ & 1265 & 69  & 2.6 & 62         \\
%     Tensor      & 192 & $64^3$ & 1265 & 128 & 2.4 & 323        \\
    
%     Matrix-free & 1536 & $96^3$ & 576  & 47  & 2.2 & 60  \\
%     Tensor      & 1536 & $96^3$ & 576  & 72  & 2.2 & 252 \\
    
%     Tensor      & 12288 & $192^3$ & 576  & 26 & 1.1 & 166 \\
%     \bottomrule
%   \end{tabular}
% \end{frame}

\begin{frame}{HPGMG: a new benchmarking proposal}
  \begin{itemize}
  \item \url{https://hpgmg.org}, hpgmg-forum@hpgmg.org mailing list
  \item SC14 BoF: Wednesday, Nov 19, 12:15pm to 1:15pm
  \item Mark Adams, Sam Williams (finite-volume), myself (finite-element), John Shalf, Brian Van Straalen, Erich Strohmeier, Rich Vuduc
  \item Implementations
    \begin{description}
    \item[Finite Volume] memory bandwidth intensive, simple data dependencies
    \item[Finite Element] compute- and cache-intensive, vectorizes
    \end{description}
  \item Full multigrid, well-defined, scale-free problem
  \item Goal: necessary and sufficient
    \begin{itemize}
    \item Every feature stressed by benchmark should be necessary for an important application
    \item Good performance on the benchmark should be sufficient for good performance on most applications
    \end{itemize}
  \end{itemize}
\end{frame}

\begin{frame}{Kiviat diagrams}
  \begin{center}
    \includegraphics[width=\textwidth]{figures/hpgmg-kiviat-20140606.png}
  \end{center}
  \begin{itemize}
  \item c/o Ian Karlin and Bert Still (LLNL)
  \end{itemize}
\end{frame}

\begin{frame}{HPGMG distinguishes networks}
  \begin{center}
    \includegraphics[width=0.5\textwidth]{figures/hpgmg-fv-20140515-dof.png}
  \end{center}
  \begin{itemize}
  \item About 1M dof/socket
  \item Peregrine and Edison have identical node architecture
  \item Peregrine has 5:1 tapered IB
  \end{itemize}
\end{frame}

\begin{frame}{Dynamic Range}
  \begin{center}
    \includegraphics[width=0.65\textwidth]{figures/hpgmgfe-edison-vesta.png}
  \end{center}
  \begin{itemize}
  \item BG/Q vectorization overloads cache, load/store: 88\% FXU, 12\% FPU
  \item Users like predictable performance across a range of problem sizes
  \item Half of all PETSc users care about strong scaling more
  \item Transient problems do not weak scale even if each step does
  \end{itemize}
\end{frame}

\begin{frame}{Where we are now: $QR$ factorization with MKL on MIC}
  \begin{figure}
    \centering
    \includegraphics[width=\textwidth]{figures/hardware/MKL-dgeqrf-MIC}
  \end{figure}
  \begin{itemize}
  \item Figure compares two CPU sockets (230W TDP) to one MIC (300W TDP plus host)
  \item Performance/Watt only breaks even at largest problem sizes
  \item $10^4 \times 10^4$ matrix takes 667 GFlops: about 2 seconds
  \item This is an $O(n^{3/2})$ operation on $n$ data
  \item MIC cannot strong scale, no more energy efficient/cost effective
  \end{itemize}
\end{frame}

\begin{frame}{Outlook}
  \begin{itemize}
  \item Memory bandwidth is a major limitation
  \item Can change algorithms to increase intensity
    \begin{itemize}
    \item Usually increases stress on cache
    \end{itemize}
  \item Optimizing for vectorization can incur large bandwidth overhead
  \item I think data motion is a more fundamental long-term concern
  \item Latency is at least as important as throughput for many applications
  \item ``hard to program'' versus ``architecture ill-suited for problem''?
  \item Performance varies with configuration
    \begin{itemize}
    \item number of tracers, number of levels, desired steps/second
    \item do not need optimality in all cases, but should degrade gracefully
    \end{itemize}
  \end{itemize}
\end{frame}

\end{document}
