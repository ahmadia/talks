% \documentclass[handout]{beamer}
\documentclass{beamer}

\mode<presentation>
{
  \usetheme{ANLBlue}
  % \usefonttheme[onlymath]{serif}
  % \usetheme{Singapore}
  % \usetheme{Warsaw}
  % \usetheme{Malmoe}
  % \useinnertheme{circles}
  % \useoutertheme{infolines}
  % \useinnertheme{rounded}

  \setbeamercovered{transparent=05}
}

\usepackage[english]{babel}
\usepackage[latin1]{inputenc}
\usepackage{alltt,listings,multirow,ulem,siunitx}
\usepackage[absolute,overlay]{textpos}
\TPGrid{1}{1}
\usepackage{pdfpages}
\usepackage{ulem}
\usepackage{multimedia}
\usepackage{multicol}
\newcommand\hmmax{0}
\newcommand\bmmax{0}
\usepackage{bm}
\usepackage{comment}
\usepackage{subcaption}

% font definitions, try \usepackage{ae} instead of the following
% three lines if you don't like this look
\usepackage{mathptmx}
\usepackage[scaled=.90]{helvet}
% \usepackage{courier}
\usepackage[T1]{fontenc}
\usepackage{tikz}
\usetikzlibrary{decorations.pathreplacing}
\usetikzlibrary{shadows,arrows,shapes.misc,shapes.arrows,shapes.multipart,arrows,decorations.pathmorphing,backgrounds,positioning,fit,petri,calc,shadows,chains,matrix}

\newcommand\vvec{\bm v}
\newcommand\bvec{\bm b}
\newcommand\bxk{\bvec_0 \times \kappa_0 \cdot \nabla}
\newcommand\delp{\nabla_\perp}

% \usepackage{pgfpages}
% \pgfpagesuselayout{4 on 1}[a4paper,landscape,border shrink=5mm]

\usepackage{JedMacros}

\newcommand{\timeR}{t_{\mathrm{R}}}
\newcommand{\timeW}{t_{\mathrm{W}}}
\newcommand{\mglevel}{\ensuremath{\ell}}
\newcommand{\mglevelcp}{\ensuremath{\mglevel_{\mathrm{cp}}}}
\newcommand{\mglevelcoarse}{\ensuremath{\mglevel_{\mathrm{coarse}}}}
\newcommand{\mglevelfine}{\ensuremath{\mglevel_{\mathrm{fine}}}}

%solution and residual
\newcommand{\vx}{\ensuremath{x}}
\newcommand{\vc}{\ensuremath{\hat{x}}}
\newcommand{\vr}{\ensuremath{r}}
\newcommand{\vb}{\ensuremath{b}}

%operators
\newcommand{\vA}{\ensuremath{A}}
\newcommand{\vP}{\ensuremath{I_H^h}}
\newcommand{\vS}{\ensuremath{S}}
\newcommand{\vR}{\ensuremath{I_h^H}}
\newcommand{\vI}{\ensuremath{\hat I_h^H}}
\newcommand{\vV}{\ensuremath{\mathbf{V}}}
\newcommand{\vF}{\ensuremath{F}}
\newcommand{\vtau}{\ensuremath{\mathbf{\tau}}}


\title{Numerical Libraries and Frameworks (PETSc)}

\author{{\bf Jed Brown} \texttt{jedbrown@mcs.anl.gov} \\
  Argonne National Lab and CU Boulder
}

% - Use the \inst command only if there are several affiliations.
% - Keep it simple, no one is interested in your street address.
% \institute
% {
%   Mathematics and Computer Science Division \\ Argonne National Laboratory
% }

\date{ENES Workshop on Exascale Technologies, 2014-03-18}

% This is only inserted into the PDF information catalog. Can be left
% out.
\subject{Talks}


% If you have a file called "university-logo-filename.xxx", where xxx
% is a graphic format that can be processed by latex or pdflatex,
% resp., then you can add a logo as follows:

% \pgfdeclareimage[height=0.5cm]{university-logo}{university-logo-filename}
% \logo{\pgfuseimage{university-logo}}



% Delete this, if you do not want the table of contents to pop up at
% the beginning of each subsection:
% \AtBeginSubsection[]
% {
% \begin{frame}<beamer>
%   \frametitle{Outline}
%   \tableofcontents[currentsection,currentsubsection]
% \end{frame}
% }

\AtBeginSection[]
{
  \begin{frame}<beamer>
    \frametitle{Outline}
    \tableofcontents[currentsection]
  \end{frame}
}

% If you wish to uncover everything in a step-wise fashion, uncomment
% the following command:

% \beamerdefaultoverlayspecification{<+->}

\begin{document}
\lstset{language=C}
\normalem

\begin{frame}
  \titlepage
\end{frame}

\begin{frame}{What can libraries offer?}
  \begin{itemize}
  \item Code reuse
    \begin{itemize}
    \item Porting/optimization to new architectures
    \item \ldots but only the part of the problem solved by the library
    \end{itemize}
  \item Easy experimentation with different methods
    \begin{itemize}
    \item via run-time options (PETSc)
    \item ``black box'' solvers are not sustainable
    \item preconditioners, linear and nonlinear accelerators, time integrators
    \end{itemize}
  \item Diagnostic and debugging support
    \begin{itemize}
    \item Convergence monitors, error estimators, adaptive controllers
    \item Compatibility checks
    \item Eigen-analysis
    \end{itemize}
  \item Communication with algorithm developers
    \begin{itemize}
    \item Precise language to describe methods
    \item Performance diagnostics
    \end{itemize}
  \item Flexible coupling algorithms: beyond ``first-order'' splitting
  \end{itemize}
\end{frame}

\begin{frame}{Library or Framework?}
  \begin{columns}
    \begin{column}{0.5\textwidth}
      \begin{block}{Library}
        \begin{itemize}
        \item Libraries provide a toolbox
        \item No assumptions about usage
        \item Any selection of libraries should be usable in combination
        \item \textit{Extensible} libraries enable user to implement/extend
        \end{itemize}
      \end{block}
    \end{column}
    \begin{column}{0.5\textwidth}
      \begin{block}{Framework}
        \begin{itemize}
        \item Rapid development within a problem class
        \item End-to-end solution provides guidance and auxiliary tools
        \item Opinionated
        \item Hard to use in combination with other Frameworks1
        \end{itemize}
      \end{block}
    \end{column}
  \end{columns}
  \begin{center}
    \uncover<2>{\alert{\Huge PETSc is a Library}}
  \end{center}
\end{frame}

\begin{frame}{Portable {\bf Extensible} Toolkit for Scientific computing}
  \begin{block}{Portable}
    \begin{itemize}
    \item Runs \emph{performantly} from laptop and iPhone to BG/Q and Titan
    \item Any compiler, any OS
    \item C, C++, Fortran 77 \& 90+, Python, MATLAB
    \item Free to everyone: BSD-style license, open development
    \end{itemize}
  \end{block}
  \begin{block}{Philosophy: Everything has a plugin architecture}
    \begin{itemize}
    \item Vectors, Matrices, Coloring/ordering/partitioning algorithms
    \item Preconditioners, Krylov accelerators, Nonlinear solvers, Time integrators
    \item Spatial discretizations/topology$^*$
    \item Example: Third party supplies matrix format and associated preconditioner, distributes
      compiled shared library.  Application user loads plugin at runtime, no source
      code in sight.
    \end{itemize}
  \end{block}
\end{frame}

\begin{frame}{Portable Extensible Toolkit for {\bf Scientific computing}}
  \begin{itemize}
  \item Computational Scientists and Engineers
    \begin{itemize}
    \item Structural mechanics, CFD, Geodynamics, Subsurface flow, Reactor engineering, Fusion
    \item Research (many countries, many agencies) and industry (oil and gas, aerospace, ABAQUS)
    \end{itemize}
  \item Algorithm Developers (iterative methods and preconditioning)
    \begin{itemize}
    \item Example: Ghysels' pipelined Krylov methods
    \end{itemize}
  \item Package Developers
    \begin{itemize}
    \item SLEPc, TAO, Libmesh, MOOSE, FEniCS, Deal.II, etc
    \end{itemize}
  \item Funding
    \begin{itemize}
    \item Department of Energy (SciDAC, ASCR, collaborations)
    \item National Science Foundation (CIG and others)
    \end{itemize}
  \item Active development team with long-term commitment
  \item Hundreds of tutorial-style examples
  \item Hyperlinked manual, examples, and manual pages for all routines
  \item Lists: \url{petsc-users@mcs.anl.gov}, \url{petsc-dev@mcs.anl.gov}
  \item Support from \url{petsc-maint@mcs.anl.gov}
\end{itemize}
\end{frame}

\begin{frame}{Solvers in climate}
  \begin{itemize}
  \item ``Pressure'' solves for semi-implicit methods
    \begin{itemize}
    \item Depends on separation between fastest wave and dynamics
    \end{itemize}
  \item Time integration for atmospheric column physics
    \begin{itemize}
    \item Currently swamped with splitting error
    \item Stiff, positivity constraints, non-smoothness (freezing)
    \end{itemize}
  \item Sea ice
    \begin{itemize}
    \item Fast elastic wave speed ($v_p \approx \SI{3}{\kilo\metre\per\second}$)
    \item Damped EVP model not converged at 120 subcycles, nor at 1200 (Lemieux at al 2012)
    \end{itemize}
  \item Land ice (Stokes and hydrostatic models with slippery bed)
    \begin{itemize}
    \item PETSc: PISM (UAF, PIK), BISICLES (LBL, Chombo), ISSM (NASA)
    \end{itemize}
  \item Improved stability for symplectic integration
  \item Accelerated spin-up (e.g., deep ocean)
    \begin{itemize}
    \item Need to model unresolved-in-time processes
    \end{itemize}
  \end{itemize}
\end{frame}

\begin{frame}
  \includegraphics[width=\textwidth]{figures/TS/CaldwellTimeStepConvergence.png} \\
  %
  c/o Peter Caldwell (LLNL)
  \begin{itemize}
  \item Models calibrated for ``efficient'' time step
  \item No longer solving the PDEs we write down
  \item Expensive to recalibrate when discretization changes
  \item Calibration eats up a big chunk of the IPCC policy timeline
  \end{itemize}
\end{frame}

\begin{frame}{Sea Ice}
  \begin{columns}
    \begin{column}{0.75\textwidth}
      {\scriptsize
        \begin{gather*}
          (\rho h \bm u)_t + \underbrace{\rho h f \bm k \times \bm u}_{\text{Coriolis}} - \underbrace{\bm \tau}_{\text{water/air}} + \underbrace{\rho g h \nabla H_d}_{\text{surface gradient}} - \nabla\cdot(\underbrace{\rho h \bm u \otimes \bm u}_{\text{convection}} - \underbrace{\sigma}_{\text{viscoplastic}}) = 0 \\
          \sigma = 2 \eta \dot\epsilon + \big[(\zeta - \eta) \trace\dot\epsilon - P/2 \big] \bm 1
        \end{gather*}}
    \end{column}
    \begin{column}{0.25\textwidth}
      \includegraphics[width=\textwidth]{figures/SeaIce/VorticesIceStrainRate}      
    \end{column}
  \end{columns}
  \begin{itemize}
  \item mildly nonsymmetric due to Coriolis (quasi-diagonal) and convection (small compared to viscous stresses)
  \item Nonlinear multigrid is less synchronous
    {\small
    \begin{tabular}{llll}
      \toprule
      Method & Nonlinear its/stage & Linear its/stage & V-cycles \\
      \midrule
      Newton-Krylov MG & 6 & 30.44 & 30.44 \\
      FAS Newton/BJacobi/SOR & 18.33 & --- & 18.33 \\
      \bottomrule
    \end{tabular}}
  \item Additive Runge-Kutta IMEX, error-based adaptivity, solver rtol $10^{-8}$
  \item Preliminary tests to 4096 cores of BG/Q and 64 fine-grid elements/process, less than 0.1 seconds/time step.
  \end{itemize}
\end{frame}

\begin{frame}{IMEX time integration in PETSc}
  \begin{itemize}
  \item Additive Runge-Kutta IMEX methods
    \begin{gather*}
      G(t,x,\dot x) = F(t,x) \\
      J_\alpha = \alpha G_{\dot x} + G_x
    \end{gather*}
    \vspace{-1em}
    \begin{itemize}
    \item User provides:
      \begin{itemize}
      \item \texttt{FormRHSFunction(ts,$t$,$x$,$F$,void *ctx);}
      \item \texttt{FormIFunction(ts,$t$,$x$,$\dot x$,$G$,void *ctx);}
      \item \texttt{FormIJacobian(ts,$t$,$x$,$\dot x$,$\alpha$,$J$,$J_{p}$,mstr,void *ctx);}
      \end{itemize}
    \item Can have $L$-stable DIRK for stiff part $G$, SSP explicit part, etc.
    \item Orders 2 through 5, embedded error estimates
    \item Dense output, hot starts for Newton
    \item More accurate methods if $G$ is linear, also Rosenbrock-W
    \item Can use preconditioner from classical ``semi-implicit'' methods
    \item FAS nonlinear solves supported
    \item Extensible adaptive controllers, can change order within a family
    \item Easy to register new methods: \code{TSARKIMEXRegister()}
    \end{itemize}
  \item Single step interface so user can have own time loop
  \item Same interface for Extrapolation IMEX, LMS IMEX (in development)
  \end{itemize}
\end{frame}

\input{slides/MonolithicOrSplit.tex}
\begin{frame}{Multi-physics coupling in PETSc}
  \begin{columns}
    \begin{column}{0.5\textwidth}
      \tikzstyle{cloud} = [draw, ellipse,fill=red!20, node distance=3cm, minimum height=2em]
      \tikzstyle{block} = [rectangle, draw, fill=blue!20, text width=5em, text centered, rounded corners, minimum height=2em]
      \begin{tikzpicture}
        \node [cloud] (momentum) {Momentum};
        \node [cloud, right of=momentum] (pressure) {Pressure};
        \node<2-> [block, opacity=0.5, fit=(momentum)(pressure), text opacity=0.8] (stokes) {Stokes};
        % ]
      \end{tikzpicture}
    \end{column}
    \begin{column}{0.5\textwidth}
      \begin{itemize}
      \item package each ``physics'' independently
      \item solve single-physics and coupled problems
      \item semi-implicit and fully implicit
      \item reuse residual and Jacobian evaluation unmodified
      \item direct solvers, fieldsplit inside multigrid, multigrid inside fieldsplit without recompilation
      \item use the best possible matrix format for each physics \\ (e.g. symmetric block size 3)
      \item matrix-free anywhere
      \item multiple levels of nesting
      \end{itemize}
    \end{column}
  \end{columns}
\end{frame}

\begin{frame}{Splitting for Multiphysics}
  \begin{equation*}
    \begin{bmatrix}
      A & B \\ C & D
    \end{bmatrix}
    \begin{bmatrix}
      V \\ P
    \end{bmatrix}
    =
    \begin{bmatrix}
\frac{\partial F^u}{\partial U} & \frac{\partial F^u}{\partial P}\\ \frac{\partial F^p}{\partial U} & \frac{\partial F^p}{\partial P}
    \end{bmatrix}
    \begin{bmatrix}
      V \\ P
    \end{bmatrix}
    =
    \begin{bmatrix}
      f \\ g
    \end{bmatrix}
  \end{equation*}
  \begin{itemize}\item Relaxation:
    \code{-pc\_fieldsplit\_type [additive,multiplicative,symmetric\_multiplicative]}
    \begin{equation*}
      \begin{bmatrix}
        A & \\  & D
      \end{bmatrix}^{-1} \qquad 
      \begin{bmatrix}
        A & \\ C & D
      \end{bmatrix}^{-1} \qquad
      \begin{bmatrix}
        A & \\  & \bm 1
      \end{bmatrix}^{-1}
      \left(
        \bm 1 -
        \begin{bmatrix}
          A & B \\ & \bm 1
        \end{bmatrix}
        \begin{bmatrix}
          A & \\ C & D
        \end{bmatrix}^{-1}
      \right)
    \end{equation*}
    \begin{itemize}
    \item Gauss-Seidel inspired, works when fields are loosely coupled
    \end{itemize}
  \item Factorization: \code{-pc\_fieldsplit\_type schur}
    \begin{align*}
      \begin{bmatrix}
        A & B \\ & S
      \end{bmatrix}^{-1}
      \begin{bmatrix}
        1 & \\ CA^{-1} & 1
      \end{bmatrix}^{-1}, \qquad
      S = D - C A^{-1} B
    \end{align*}
    \begin{itemize}
    \item robust (exact factorization), can often drop lower block
    \item how to precondition $S$ which is usually dense?
      \begin{itemize}
      \item interpret as differential operators, use approximate commutators
      \end{itemize}
    \end{itemize}
  \item ``Composable Linear Solvers for Multiphysics'' ISPDC 2012
  \end{itemize}
\end{frame}

\begin{frame}{Eigen-analysis plugin for solver design}
  Hydrostatic ice flow (nonlinear rheology and slip conditions)
  \begin{align}\label{eq:momentum}
    - \nabla \left[ \eta
      \begin{pmatrix}
        4 u_x + 2 v_y & u_y + v_x & u_z \\
        u_y + v_x & 2 u_x + 4 v_y & v_z
      \end{pmatrix} \right] + \rho g \nabla s & = 0,
  \end{align}
  \begin{itemize}
  \item Many solvers converge easily with no-slip/frozen bed, more difficult for slippery bed (ISMIP HOM test C)
  \item Geometric MG is good: $\lambda \in [0.805, 1]$ (SISC 2013)
  \end{itemize}
  % GAMG: ./ex48 -M 10 -P 8 -da_refine 1 -thi_mat_type aij -thi_hom C -dll_append ~/petsc-eig/mpich-opt/lib/libpetsc-eig.so -ksp_plugin eig -eig_type preconditioned -eig_eps_nev 10 -eig_eps_smallest_real -eig_view_vectors_vtk -eig_st_ksp_type gmres -eig_st_ksp_rtol 1e-9 -eig_eps_monitor_lg_all -eig_eps_view -pc_type gamg
  % Eigenvalue  0 (error): 0.0268052+0i (2.34383e-09)
  % Eigenvalue  1 (error): 0.0408511+0i (9.28564e-10)
  % Eigenvalue  2 (error): 0.0431757+0i (7.35697e-10)
  % Eigenvalue  3 (error): 0.0447336+0i (6.78016e-09)
  % Eigenvalue  4 (error): 0.0490315+0i (8.74661e-09)
  % Eigenvalue  5 (error): 0.0539488+0i (9.67847e-10)
  % Eigenvalue  6 (error): 0.055815+0i (1.7793e-09)
  % Eigenvalue  7 (error): 0.0598606+0i (1.92014e-09)
  % Eigenvalue  8 (error): 0.06518+0i (3.2315e-09)
  % Eigenvalue  9 (error): 0.0669961+0i (2.8736e-09)
  \vspace{-1ex}
  \begin{figure}
    \centering
    \begin{subfigure}{0.4\textwidth}
      \centering
      \includegraphics[width=\textwidth]{figures/THI/EigenGAMG/visit0000.png}
      \caption{$\lambda_0 = 0.0268$}
    \end{subfigure}
    \begin{subfigure}{0.4\textwidth}
      \centering
      \includegraphics[width=\textwidth]{figures/THI/EigenGAMG/visit0001.png}
      \caption{$\lambda_1 = 0.0409$}
    \end{subfigure}
    % \caption{Smallest two eigenpairs for smoothed aggregation with only translational modes (but no rotational modes).}
  \end{figure}
\end{frame}


\begin{frame}{Implicit Runge-Kutta for advection}
  \begin{table}
    \centering
    \caption{Total number of iterations (communications or accesses of $J$) to solve linear advection to $t=1$ on a $1024$-point grid using point-block Jacobi preconditioning of implicit Runge-Kutta matrix.
      The relative algebraic solver tolerance is $10^{-8}$.}\label{tab:irk-advection}
    \begin{tabular}{lrrr}
      \toprule
      Family & Stages & Order & Iterations \\
      \midrule
      Crank-Nicolson/Gauss & 1 & 2 & 3627 \\
      Gauss & 2 & 4 & 2560 \\
      Gauss & 4 & 8 & 1735 \\
      Gauss & 8 & 16 & 1442 \\
      \bottomrule
    \end{tabular}
  \end{table}
  \begin{itemize}
  \item Naive centered-difference discretization
  \end{itemize}
\end{frame}

\begin{frame}{A case for run-time configuration}
  \begin{itemize}
  \item Simple build process
  \item Complete test suite without recompilation
  \item Cleaner provenance
    \begin{itemize}
    \item Only need run-time configuration
    \item No recompiles, only one binary to keep track of
    \item Consistency checks in one place
    \end{itemize}
  \item Simplified analysis/uncertainty quantification
    \begin{itemize}
    \item More algorithms accessible
    \end{itemize}
  \item More automated calibration
  \item Interface granularity is key to performance
  \end{itemize}
\end{frame}

\begin{frame}{Outlook}
  \begin{itemize}
  \item PETSc: flexible, extensible, unintrusive
    \begin{itemize}
    \item \url{http://mcs.anl.gov/petsc}
    \end{itemize}
  \item Verification (converging the equations) encourages mathematicians
  \item Climate model components \emph{should} become more library-like
    \begin{itemize}
    \item Remove assumptions about environment
    \item Improved modularity
    \item Interfaces for configuration/calibration
    \item Remove global variables (Fortran module variables)
    \end{itemize}
  \item Tools need to make hard problems possible
    \begin{itemize}
    \item Already many tools to make easy problems elegant
    \item Ease of extending (versus DSLs/compilers)
    \end{itemize}
  \item Strong-scaling necessity: ruthlessly shorten critical path
    \begin{itemize}
    \item $2\times$ increase in resolution requires at least $2\times$ more steps
    \item At fixed turn-around time, need twice as many steps/second
    \item Algorithmic optimality is crucial
    \end{itemize}
  \end{itemize}
\end{frame}

\end{document}
