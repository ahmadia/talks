\begin{frame}{What makes a nonlinear algebraic problem difficult?}
  \begin{itemize}
  \item Ill-conditioning
    \begin{itemize}
    \item For a second order elliptic problem, condition number $\sim \bigO(h^{-2})$.
    \item Anisotropy, and coefficient structure make the issue worse
    \item Iterative algorithms with local corrections cannot fix these asymptotics
    \item Eventual convergence guaranteed for many methods (in exact arithmetic)
    \item Practical convergence too slow, worse in parallel
    \end{itemize}
  \item Non-locality
    \begin{itemize}
    \item Non-symmetric processes
    \item Multiscale coefficients/anisotropy
    \end{itemize}
  \item Nonlinearity (especially non-smoothness)
    \begin{itemize}
    \item Plasticity, contact, shocks, phase change, bifurcations
    \item Usually local (in some sense), but can have long-range effects
    \item Methods stop converging as parameters change
    \end{itemize}
  \item<2> Nonlinearity interacts with ill-conditioning
  \end{itemize}
\end{frame}
