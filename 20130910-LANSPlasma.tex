% \documentclass[handout]{beamer}
\documentclass{beamer}

\mode<presentation>
{
  \usetheme{ANLBlue}
  % \usefonttheme[onlymath]{serif}
  % \usetheme{Singapore}
  % \usetheme{Warsaw}
  % \usetheme{Malmoe}
  % \useinnertheme{circles}
  % \useoutertheme{infolines}
  % \useinnertheme{rounded}

  \setbeamercovered{transparent=20}
}

\usepackage[english]{babel}
\usepackage[latin1]{inputenc}
\usepackage{alltt,listings,multirow,ulem,siunitx}
\usepackage[absolute,overlay]{textpos}
\TPGrid{1}{1}
\usepackage{pdfpages}
\usepackage{ulem}
\usepackage{multimedia}
\usepackage{multicol}
\newcommand\hmmax{0}
\newcommand\bmmax{0}
\usepackage{bm}
\usepackage{comment}

% font definitions, try \usepackage{ae} instead of the following
% three lines if you don't like this look
\usepackage{mathptmx}
\usepackage[scaled=.90]{helvet}
% \usepackage{courier}
\usepackage[T1]{fontenc}
\usepackage{tikz}
\usetikzlibrary{decorations.pathreplacing}
\usetikzlibrary{shadows,arrows,shapes.misc,shapes.arrows,shapes.multipart,arrows,decorations.pathmorphing,backgrounds,positioning,fit,petri,calc,shadows,chains,matrix}

\newcommand\vvec{\bm v}
\newcommand\bvec{\bm b}
\newcommand\bxk{\bvec_0 \times \kappa_0 \cdot \nabla}
\newcommand\delp{\nabla_\perp}

% \usepackage{pgfpages}
% \pgfpagesuselayout{4 on 1}[a4paper,landscape,border shrink=5mm]

\usepackage{JedMacros}

\newcommand{\timeR}{t_{\mathrm{R}}}
\newcommand{\timeW}{t_{\mathrm{W}}}
\newcommand{\mglevel}{\ensuremath{\ell}}
\newcommand{\mglevelcp}{\ensuremath{\mglevel_{\mathrm{cp}}}}
\newcommand{\mglevelcoarse}{\ensuremath{\mglevel_{\mathrm{coarse}}}}
\newcommand{\mglevelfine}{\ensuremath{\mglevel_{\mathrm{fine}}}}

%solution and residual
\newcommand{\vx}{\ensuremath{x}}
\newcommand{\vc}{\ensuremath{\hat{x}}}
\newcommand{\vr}{\ensuremath{r}}
\newcommand{\vb}{\ensuremath{b}}

%operators
\newcommand{\vA}{\ensuremath{A}}
\newcommand{\vP}{\ensuremath{I_H^h}}
\newcommand{\vS}{\ensuremath{S}}
\newcommand{\vR}{\ensuremath{I_h^H}}
\newcommand{\vI}{\ensuremath{\hat I_h^H}}
\newcommand{\vV}{\ensuremath{\mathbf{V}}}
\newcommand{\vF}{\ensuremath{F}}
\newcommand{\vtau}{\ensuremath{\mathbf{\tau}}}


\title{Time integration and solver challenges in tokamak edge plasma simulation}
\author{{\bf Jed Brown} \\
Peter Brune, Emil Constantinescu, \\
Debojyoti Ghosh, Lois Curfman McInnes \\
\texttt{\{jedbrown,brune,emconsta,ghosh,curfman\}@mcs.anl.gov}
}

% - Use the \inst command only if there are several affiliations.
% - Keep it simple, no one is interested in your street address.
\institute
{
  Mathematics and Computer Science Division \\ Argonne National Laboratory
}

\date{LANS seminar, 2013-09-10}

% This is only inserted into the PDF information catalog. Can be left
% out.
\subject{Talks}


% If you have a file called "university-logo-filename.xxx", where xxx
% is a graphic format that can be processed by latex or pdflatex,
% resp., then you can add a logo as follows:

% \pgfdeclareimage[height=0.5cm]{university-logo}{university-logo-filename}
% \logo{\pgfuseimage{university-logo}}



% Delete this, if you do not want the table of contents to pop up at
% the beginning of each subsection:
% \AtBeginSubsection[]
% {
% \begin{frame}<beamer>
%   \frametitle{Outline}
%   \tableofcontents[currentsection,currentsubsection]
% \end{frame}
% }

\AtBeginSection[]
{
  \begin{frame}<beamer>
    \frametitle{Outline}
    \tableofcontents[currentsection]
  \end{frame}
}

% If you wish to uncover everything in a step-wise fashion, uncomment
% the following command:

% \beamerdefaultoverlayspecification{<+->}

\begin{document}
\lstset{language=C}
\normalem

\begin{frame}
  \titlepage
\end{frame}

\section{Tokamaks and Edge Localized Modes}
\begin{frame}{Tokamaks and BOUT++}
  \begin{columns}
    \begin{column}{.5\textwidth}
      \includegraphics[width=\textwidth]{figures/Tokamak/XPointTokamakCartoon.jpg} \\
      {\scriptsize c/o \url{http://efda.org}}
    \end{column}
    \begin{column}{.5\textwidth}
      \begin{itemize}
      \item Strong magnetic field in toroidal direction (FFT), weak spiral
      \item $\beta = \frac p p_{\text{mag}}$, high $\beta$ needed for efficient operation
      \item Ions and electrons mostly confined within flux surfaces
      \end{itemize}
    \end{column}
  \end{columns}
  \begin{itemize}
  \item BOUT++ (Dudson, Umansky, and others): finite difference Framework
  \item non-orthogonal grids with complicated topology
  \item C++ operator-overloaded interface
  \item Libraries: FFTW, NetCDF, Sundials, PETSc
  \end{itemize}
\end{frame}

\begin{frame}{Edge Localized Modes (ELMs)}
  \begin{figure}
    \centering
    \includegraphics[width=.5\textwidth]{figures/Tokamak/LiangHModeLMode.png}
    \includegraphics[width=.45\textwidth]{figures/Tokamak/LiangPeelingBallooning.png}
  \end{figure}
  \begin{itemize}
  \item ELMs at ITER scale could produce \SI{10}{\mega\joule\per\meter\squared}
  \item Current materials can only tolerate \SI{.5}{\mega\joule\per\meter\squared}
  \end{itemize}
\end{frame}

\section{Preconditioning}
\begin{frame}{Waves in MHD}
  \begin{itemize}
  \item Interested in peeling-ballooning modes with time scale of \SI{100}{\micro\second} to milliseconds
  \item Shear Alfv\'en waves travel along magnetic field lines: $v_A \approx \SI{1e7}{\metre\per\second}$
  \item Explicit CFL from $v_A$ would be about \SI{1e-8}{\second}
  \item Parallel electron heat conduction is similarly fast
  \item Acoustic wave velocity \SI{1e5}{\metre\per\second}: mildly stiff
  \end{itemize}
\end{frame}

\begin{frame}{Preconditioning approach (Chac\'on)}
  \begin{columns}
    \begin{column}{0.5\textwidth}
      \begin{itemize}
      \item $\rho$ density
      \item $T_e$ electron temperature
      \item $T_i$ ion temperature
      \end{itemize}       
    \end{column}
    \begin{column}{0.5\textwidth}
      \begin{itemize}
      \item $v_\parallel$  parallel velocity
      \item $A_\parallel$ parallel vector potential
      \item $U$ vorticity
      \end{itemize}
    \end{column}
  \end{columns}
  \begin{itemize}
  \item After dropping slow terms
    \begin{equation}
      \widetilde{\bm 1 - \gamma J} = \begin{pmatrix} \text{block} &  & B_{\rho,U} \\
          & \text{diagonal} & \vdots \\
         C_{U\rho} & \dotsb & D_{UU}
       \end{pmatrix}
    \end{equation}
  \item Eliminate all but vorticity $U$ and approximate Schur complement using
    \begin{equation*}
      \tilde S = \bm 1 - \gamma^2 \frac{B_0^2}{\rho_0} \partial_\parallel \nabla_\perp^2 \partial_\parallel \nabla_\perp^{-2}
    \end{equation*}
  \item Approximate commutator
    \begin{equation*}
      \partial_\parallel \nabla_\perp^2 \partial_\parallel \nabla_\perp^{-2} \approx \partial_\parallel^2 - \frac{2 \partial_\parallel^2 (RB_\theta)}{RB_\theta}
    \end{equation*}
  \end{itemize}
\end{frame}

\begin{frame}{DAE formulation}
  \begin{itemize}
  \item 3-field model: pressure $P$, magnetic flux $\psi$, vorticity $U$
  \item Electrostatic potential $\phi$ eliminated
  \end{itemize}
  \begin{equation*}
    J =
    \left[ \begin{array}{c|c|c}
        \color{blue}{-\vvec_E\cdot\nabla} & 0 & \left[\bvec_0\times\nabla\left(P_0 + \color{blue}{P}\right)\cdot\nabla\right]\nabla_\perp^{-2} \\
        \hline
        0 & \color{blue}{-\vvec_E\cdot\nabla} & \left(\bvec_0\cdot\nabla\right)\nabla_\perp^{-2}  \\
        \hline
        2\bxk  & -\frac{B_0^2}{\mu_0\rho}\left(\bvec_0 \color{blue}{-\bvec_0\times\nabla\psi}\right)\cdot\nabla\delp  & \color{blue}{-\vvec_E\cdot\nabla} \\
        & + \frac{B_0^2}{\rho}\left[\bvec_0\times\nabla\left(\frac{J_{||0}}{B_0}\right)\right]\cdot\nabla & \\
        & + \color{blue}{\frac{B_0^2}{\mu_0\rho}\nabla\left(\delp\psi\right)\cdot\left(\bvec_0\times\nabla\right)} &
      \end{array}\right]
  \end{equation*}
  \begin{itemize}
  \item $\nabla_\perp^{-1}$ consists of separate solves on subcommunicators
  \item DAE approach: add $\phi$ as additional variable
  \end{itemize}
\end{frame}

\section{Time Integration}
\begin{frame}{Trade-offs in time integration}
  \begin{itemize}
  \item Properties
    \begin{itemize}
    \item Nonlinear stability (e.g., positivity preservation)
    \item Stability along imaginary axis
    \item $L$-stability (damping at infinity)
    \item Implicitness and reuse
    \end{itemize}
  \item What is expensive?
    \begin{itemize}
    \item Function evaluation
    \item Operator assembly/preconditioner setup
      \begin{itemize}
      \item How much can be reused for how long?
      \end{itemize}
    \item Implicit solves
      \begin{itemize}
      \item Can we find better solver algorithm?
      \item More effort in setup?
      \end{itemize}
    \end{itemize}
  \item What is ``convergence''?
    \begin{itemize}
    \item Wave propagation: implicitness useless for convergence \emph{in a norm}
    \item Non-norm functionals could be robust
    \end{itemize}
  \end{itemize}
\end{frame}

\begin{frame}{Reusing implicit solver setup}
  \begin{itemize}
  \item Linearization
  \item MG interpolants
  \item Lagged preconditioner
  \item Modified Newton
  \item Quasi-Newton
  \item IMEX with linear implicit part
  \item Rosenbrock/W
  \end{itemize}
\end{frame}

\begin{frame}{IMEX time integration in PETSc}
  \begin{itemize}
  \item Additive Runge-Kutta IMEX methods
    \begin{gather*}
      G(t,x,\dot x) = F(t,x) \\
      J_\alpha = \alpha G_{\dot x} + G_x
    \end{gather*}
    \vspace{-1em}
    \begin{itemize}
    \item User provides:
      \begin{itemize}
      \item \texttt{FormRHSFunction(ts,$t$,$x$,$F$,void *ctx);}
      \item \texttt{FormIFunction(ts,$t$,$x$,$\dot x$,$G$,void *ctx);}
      \item \texttt{FormIJacobian(ts,$t$,$x$,$\dot x$,$\alpha$,$J$,$J_{p}$,mstr,void *ctx);}
      \end{itemize}
    \item Can have $L$-stable DIRK for stiff part $G$, SSP explicit part, etc.
    \item Orders 2 through 5, embedded error estimates
    \item Dense output, hot starts for Newton
    \item More accurate methods if $G$ is linear, also Rosenbrock-W
    \item Can use preconditioner from classical ``semi-implicit'' methods
    \item FAS nonlinear solves supported
    \item Extensible adaptive controllers, can change order within a family
    \item Easy to register new methods: \code{TSARKIMEXRegister()}
    \end{itemize}
  \item Single step interface so user can have own time loop
  \item Same interface for Extrapolation IMEX, LMS IMEX (in development)
  \end{itemize}
\end{frame}


\begin{frame}[fragile]{Time integration method design}
  \begin{figure}
    \centering
    \includegraphics[width=.8\textwidth]{figures/TS/EmilMethodDesignFeatures.png}
  \end{figure}
  \begin{itemize}
  \item Select order, number of stages, required properties
  \item Optimize properties like SSP coefficient, accuracy, or linear stability
  \item \cverb|TSARKIMEXRegister("my-method", ...coefficients...)|
  \item \cverb|-ts_type arkimex -ts_arkimex_type my-method|
  \end{itemize}
\end{frame}

\begin{frame}{Example: Additive Runge-Kutta design}
  \begin{itemize}
  \item 3-stage, second order, $L$-stable implicit part
  \item one-parameter family of solutions
  \end{itemize}
  \begin{description}
  \item[ARK2c] Maximize SSP coefficient
  \item[ARK2E] Minimize leading error coefficient
  \end{description}
  \begin{figure}
    \centering
    \includegraphics[width=0.55\textwidth]{figures/TS/ssp_ark_poster.png}
    \includegraphics[width=0.49\textwidth]{figures/TS/Stability_ARK2E_ARK2C.pdf}
  \end{figure}
\end{frame}

\input{slides/PETSc/TSMethods.tex}

\begin{frame}{Adaptive controllers}
  \begin{itemize}
  \item ``Stiff'' waves are not stiff if one wants to converge \emph{in a norm}
    \begin{itemize}
    \item MHD has many waves, speeds can be degenerate
    \end{itemize}
  \item PETSc integrators provide embedded methods to estimate errors
  \item Automatic controllers optimize local truncation error and nonlinear solve cost
  \item User can register custom controllers
  \item Use a priori knowledge of the physics, robust functionals
  \item Choose from list of methods, choose next step size
  \end{itemize}
\end{frame}

\section{Nonlinear solvers}
\begin{frame}{Do we need nonlinear solvers?}
  \begin{itemize}
  \item Simulations start nearly-linear, strong nonlinearities take time to develop
  \item AMG setup is currently expensive
    \begin{itemize}
    \item Could improve incremental setup cost, but numeric $R A P$ takes most time
    \item Low-rank quasi-Newton lagging?
    \end{itemize}
  \item Nonlinear multigrid and domain decomposition
    \begin{itemize}
    \item ASPIN (left-preconditioned nonlinear Schwarz), also right-preconditioned
    \item Full Approximation Scheme with linear or nonlinear smoothers
    \item More intrusive, but freakishly efficient for difficult problems
    \end{itemize}
  \item Nonlinear GMRES, Anderson mixing, nonlinear CG
    \begin{itemize}
    \item Accelerator for nonlinear preconditioning
    \item Good alternative to matrix-free finite differencing
    \item More robust line search possible: operates in reduced basis
    \end{itemize}
  \end{itemize}
\end{frame}

% \begin{frame}{Nonlinear methods}
%   \begin{itemize}
%   \item Global linearization (NewtonLS, NewtonTR)
%     \begin{itemize}
%     \item Preconditioning libraries for assembled matrices
%     \item Low arithmetic intensity
%     \end{itemize}
%   \item Quasi-Newton
%     \begin{itemize}
%     \item Build low-rank updates to Jacobian inverse
%     \item Brown and Brune, ``Low-rank quasi-Newton updates for robust Jacobian lagging in Newton-type methods'', ANS MC13.
%     \end{itemize}
%   \item Nonlinear multigrid and domain decomposition
%     \begin{itemize}
%     \item ASPIN (left-preconditioned nonlinear Schwarz), also right-preconditioned
%     \item Full Approximation Scheme with linear or nonlinear smoothers
%     \item More intrusive, but freakishly efficient for difficult problems
%     \end{itemize}
%   \item Nonlinear GMRES, Anderson mixing, nonlinear CG
%     \begin{itemize}
%     \item Accelerator for nonlinear preconditioning
%     \item Good alternative to matrix-free finite differencing
%     \item More robust line search possible: operates in reduced basis
%     \end{itemize}
%   \end{itemize}
% \end{frame}
\begin{frame}
  \includegraphics[width=\textwidth]{figures/BruneNGMRESFAS2.png}
\end{frame}

\input{slides/MonolithicOrSplit.tex}
% \input{slides/PETSc/LocalSpaces.tex}

\section{Turbulence and multiscale time integration}
\begin{frame}{Microturbulence}
  \begin{itemize}
  \item Radial transport is dominated by turbulence
  \item Magnetic field causes band gap in edge
  \item Patrick Sanan has implemented HMM and FLAVORS integrators
  \item Perhaps we can step over fast time scale?
  \end{itemize}
  \includegraphics[width=0.8\textwidth]{figures/Sanan.png}
\end{frame}

% \section{Comments on performance}
% \input{slides/JFNKBottlenecks.tex}
% \begin{frame}{Scalability Warning}
  \begin{quote}\Large \centering
    The easiest way to make software scalable \\
    is to make it sequentially inefficient. \\
    (Gropp 1999)
  \end{quote}

  \begin{itemize}
  \item We really want \emph{efficient} software
  \item Need a performance model
    \begin{itemize}
    \item memory bandwidth and latency
    \item algorithmically critical operations (\eg dot products, scatters)
    \item floating point unit
    \end{itemize}
  \item Scalability shows marginal benefit of adding more cores, nothing more
  \item Constants hidden in the choice of algorithm
  \item Constants hidden in implementation
  \end{itemize}
\end{frame}

% \begin{frame}{Performance of assembled versus unassembled}
  \vspace{0.5ex}
  \includegraphics[width=.9\textwidth]{figures/TensorVsAssembly} \\
  \begin{itemize}
  \item High order Jacobian stored unassembled using coefficients at quadrature points, can use local AD
  \item Choose approximation order at run-time, independent for each field
  \item Precondition high order using assembled lowest order method
  \item Implementation $> 70\%$ of FPU peak, SpMV bandwidth wall $< 4\%$
  \end{itemize}
\end{frame}

% \begin{frame}{Hardware Arithmetic Intensity}
  \begin{tabular}{lc}
    \toprule
    Operation                         & Arithmetic Intensity (flops/B) \\
    \midrule
    Sparse matrix-vector product      & 1/6                  \\
    Dense matrix-vector product       & 1/4                  \\
    Unassembled matrix-vector product, residual & $\gtrsim 8$          \\
    %High-order residual evaluation    & $\gtrsim 5$                \\
    \bottomrule
  \end{tabular}

  \bigskip

  \begin{tabular}{lrrr}
    \toprule
    Processor & \textsc{stream} Triad (GB/s) & Peak (GF/s) & Balance (F/B) \\
    \midrule
    E5-2680 8-core      & 38   & 173  & 4.5 \\ % http://www.cs.virginia.edu/stream/stream_mail/2013/0001.html
    E5-2695v2 12-core & 45 & 230 & 5.2 \\ % Edison 460 GF and 89 GB/s Triad
    E5-2699v3 18-core & 60 & 660 & 11 \\
    % Magny Cours 16-core & 49   & 281  & 5.7 \\
    Blue Gene/Q node    & 29.3   & 205  & 7 \\ % Morozov's best Triad (https://hpgmg.org/lists/archives/hpgmg-forum/2014-June/000071.html)
    % Tesla M2090         & 120  & 665  & 5.5 \\
    Kepler K20Xm        & 160 & 1310 & 8.2 \\ % http://www.elekslabs.com/2012/11/nvidia-tesla-k20-benchmark-facts.html
    Xeon Phi SE10P      & 161 & 1060 & 6.6 \\ % http://www.cs.virginia.edu/stream/stream_mail/2013/0002.html
    % http://blogs.utexas.edu/jdm4372/tag/xeon-phi/
    \midrule
    KNL (DRAM) & 100 & 3000 & 30 \\
    KNL (MCDRAM) & 500 & 3000 & 6 \\
    \bottomrule
  \end{tabular}
\end{frame}


\begin{frame}{Outlook}
  \begin{itemize}
  \item DAE formulation preferable
  \item Linearly implicit or nonlinearly implicit with lagging?
  \item MG setup cost for ``easy'' solves
  \item Consider full MG in strongly nonlinear regime
  \item Reconsider value of spectral in toroidal direction (eventual scalability limit)
  \item Questions about gyrokinetics and full kinetics
  \item More emphasis on multiscale models
  \end{itemize}
\end{frame}

\end{document}
