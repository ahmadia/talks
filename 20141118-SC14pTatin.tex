% \documentclass[handout]{beamer}
\documentclass{beamer}
\mode<presentation>
{
  \usetheme{ANLBlue}
  % \usefonttheme[onlymath]{serif}
  % \usetheme{Singapore}
  % \usetheme{Warsaw}
  % \usetheme{Malmoe}
  % \useinnertheme{circles}
  % \useoutertheme{infolines}
  % \useinnertheme{rounded}

  \setbeamercovered{transparent=20}
}

\usepackage[english]{babel}
\usepackage[latin1]{inputenc}
%\usepackage{media9}
\usepackage{alltt,listings,multirow,ulem,siunitx}
\usepackage[absolute,overlay]{textpos}
\TPGrid{1}{1}
\usepackage{pdfpages}
\usepackage{ulem}
\usepackage{multimedia}
\usepackage{multicol}
\newcommand\hmmax{0}
\newcommand\bmmax{0}
\usepackage{bm}
\usepackage{comment}
\usepackage{subcaption}

% font definitions, try \usepackage{ae} instead of the following
% three lines if you don't like this look
\usepackage{mathptmx}
\usepackage[scaled=.90]{helvet}
% \usepackage{courier}
\usepackage[T1]{fontenc}
\usepackage{tikz}
\usetikzlibrary{decorations.pathreplacing}
\usetikzlibrary{shadows,arrows,shapes.misc,shapes.arrows,shapes.multipart,arrows,decorations.pathmorphing,backgrounds,positioning,fit,petri,calc,shadows,chains,matrix}

\newcommand\vvec{\bm v}
\newcommand\bvec{\bm b}
\newcommand\bxk{\bvec_0 \times \kappa_0 \cdot \nabla}
\newcommand\delp{\nabla_\perp}

% \usepackage{pgfpages}
% \pgfpagesuselayout{4 on 1}[a4paper,landscape,border shrink=5mm]

\usepackage{JedMacros}

\newcommand{\timeR}{t_{\mathrm{R}}}
\newcommand{\timeW}{t_{\mathrm{W}}}
\newcommand{\mglevel}{\ensuremath{\ell}}
\newcommand{\mglevelcp}{\ensuremath{\mglevel_{\mathrm{cp}}}}
\newcommand{\mglevelcoarse}{\ensuremath{\mglevel_{\mathrm{coarse}}}}
\newcommand{\mglevelfine}{\ensuremath{\mglevel_{\mathrm{fine}}}}

%solution and residual
\newcommand{\vx}{\ensuremath{x}}
\newcommand{\vc}{\ensuremath{\hat{x}}}
\newcommand{\vr}{\ensuremath{r}}
\newcommand{\vb}{\ensuremath{b}}

%operators
\newcommand{\vA}{\ensuremath{A}}
\newcommand{\vP}{\ensuremath{I_H^h}}
\newcommand{\vS}{\ensuremath{S}}
\newcommand{\vR}{\ensuremath{I_h^H}}
\newcommand{\vI}{\ensuremath{\hat I_h^H}}
\newcommand{\vV}{\ensuremath{\mathbf{V}}}
\newcommand{\vF}{\ensuremath{F}}
\newcommand{\vtau}{\ensuremath{\mathbf{\tau}}}


\title{\textbf{pTatin3d}: High-performance Methods for Long-Term Lithospheric Dynamics}
\author{Dave May, ETH Z\"urich\\
{\bf Jed Brown}, ANL \& CU Boulder\\
Laetitia Le Pourhiet, UPMC}

% - Use the \inst command only if there are several affiliations.
% - Keep it simple, no one is interested in your street address.
% \institute
% {
%   Mathematics and Computer Science Division \\ Argonne National Laboratory
% }

\date{SC14, 2014-11-18 \\[1em]
This talk: \url{http://59A2.org/files/20141118-SC14pTatin.pdf} \\
\includegraphics[width=0.2\textwidth]{figures/logos/ETHZ.png} \quad
\includegraphics[width=0.2\textwidth]{figures/logos/AnlLogo2t.png} \quad
\includegraphics[width=0.3\textwidth]{figures/logos/CUBoulder.png} \quad
\includegraphics[width=0.2\textwidth]{figures/logos/UMPC.png}
}

% This is only inserted into the PDF information catalog. Can be left
% out.
\subject{Talks}


% If you have a file called "university-logo-filename.xxx", where xxx
% is a graphic format that can be processed by latex or pdflatex,
% resp., then you can add a logo as follows:

% \pgfdeclareimage[height=0.5cm]{university-logo}{university-logo-filename}
% \logo{\pgfuseimage{university-logo}}



% Delete this, if you do not want the table of contents to pop up at
% the beginning of each subsection:
% \AtBeginSubsection[]
% {
% \begin{frame}<beamer>
%   \frametitle{Outline}
%   \tableofcontents[currentsection,currentsubsection]
% \end{frame}
% }

% \AtBeginSection[]
% {
%   \begin{frame}<beamer>
%     \frametitle{Outline}
%     \tableofcontents[currentsection]
%   \end{frame}
% }

% If you wish to uncover everything in a step-wise fashion, uncomment
% the following command:

% \beamerdefaultoverlayspecification{<+->}

\begin{document}
\lstset{language=C}
\normalem

\begin{frame}
  \titlepage
\end{frame}

\begin{frame}{Computational geodynamics}
  \begin{center}
    \includegraphics[width=\textwidth]{figures/davemay/MayGeodynamicsTimeline.png}
  \end{center}
  \vspace{-1ex}
  {\scriptsize Figure adapted from Taras Gerya}
\end{frame}

\begin{frame}{Regional scale geodynamic processes}
  \begin{center}
    \includegraphics[width=.85\textwidth]{figures/davemay/WPRegionalGeodynamicsProcesses.png}
  \end{center}
  \begin{columns}
    \begin{column}{0.5\textwidth}
      \begin{itemize}
      \item Buoyancy and topography drive flow
      \item Large variation in length scales
      \item Small scales influence large scales
      \end{itemize}
    \end{column}
    \begin{column}{0.5\textwidth}
      \begin{itemize}
      \item Complex rheology (research)
      \item Material failure (faults)
      \item Post-failure deformation
      \item Thermomechanically coupled
      \end{itemize}
    \end{column}
  \end{columns}
\end{frame}

\begin{frame}{Geology is complicated}
  \begin{columns}
    \begin{column}{0.65\textwidth}
      Zagros Mountains \\
      \includegraphics[width=\textwidth]{figures/davemay/MayZagrosMountains.png} \\
      \vspace{-1ex} {\scriptsize [Yamato et al (2011)]}
% @article{yamato2011dynamic,
%  title={Dynamic constraints on the crustal-scale rheology of the Zagros fold belt, Iran},
%  author={Yamato, Philippe and Kaus, Boris JP and Mouthereau, Fr{\'e}d{\'e}ric and Castelltort, S{\'e}bastien},
%  journal={Geology},
%  volume={39},
%  number={9},
%  pages={815--818},
%  year={2011},
%  publisher={Geological Society of America}
% }
    \end{column}
    \begin{column}{0.35\textwidth}
      European Alps \\
      \includegraphics[width=\textwidth]{figures/davemay/Schmidt2004EuropeanAlps.png}
    \end{column}
  \end{columns}
  \begin{columns}
    \begin{column}{0.6\textwidth}
      \begin{itemize}
      \item Ductile folding
      \item Discontinuous material properties
      \end{itemize}
    \end{column}
    \begin{column}{0.4\textwidth}
      \begin{itemize}
      \item Inherently 3D
      \item Faulting
      \end{itemize}
    \end{column}
  \end{columns}
\end{frame}

\begin{frame}{Continental rifting}
  %\movie[width=\textwidth,height=0.66\textwidth,autostart,poster,showcontrols=false,loop]{Oblique Rifting Video}{dstrong2_truc.mov}
  \movie[width=\textwidth,height=0.66\textwidth,autostart,poster,showcontrols=false,loop]{Rifting Video}{LaetiSC14Rifting.mp4}
\end{frame}

\begin{frame}{Continuum modeling}
  %Solve for velocity $\bm u$, pressure $p$, temperature $T$, materials
  %$\rho_i$
  \begin{itemize}
  \item Something we trust: \textbf{Conservation}
    \begin{align*}
      -\nabla\cdot \big[ \eta D \bm u - p \bm I \big] &= \rho \bm g & \text{momentum} \\
      \nabla \cdot \bm u &= 0 & \text{mass} \\
      \frac{D T}{D t} - \nabla\cdot (\kappa \nabla T) &= Q & \text{energy} \\
      \frac{D \Phi_i}{D t} &= 0 & \text{composition}
    \end{align*}
    \begin{itemize}
    \item $D \bm u = \frac 1 2 \big[ \nabla \bm u + (\nabla \bm u)^T \big]$
    \item Non-Newtonian Stokes, high-contrast coefficients $10^{10}$
    \item Boussinesq approximation, high Rayleigh number, zero Reynolds
    \item Free surface, near hydrostatic balance
    \end{itemize}
  \item Something we don't: \textbf{Constitutive models}
    \begin{itemize}
    \item $\eta(D\bm u,p,T,\Phi_i)$ shear viscosity---viscoplastic, non-smooth
      \begin{itemize}
      \item von Mises, Drucker-Prager
      \end{itemize}
    \item $\rho(p,T,\Phi_i)$ density
    \end{itemize}
  \end{itemize}
\end{frame}

\begin{frame}{Spatial discretization}
  \begin{textblock}{0.27}[1,1](1,0.95)
    \includegraphics[width=\textwidth]{figures/davemay/MayMaterialPointMethod.png} \\
    \includegraphics[width=\textwidth]{figures/davemay/MayMaterialPointMethodL2Proj.png}
  \end{textblock}
  \begin{itemize}
  \item Requirements
    \begin{itemize}
    \item Stable for resolved and under-resolved features
    \item Accurate for resolved scales
    \end{itemize}
  \item $Q_2-P_1^{\text{disc}}$ finite elements
    \begin{itemize}
    \item Linear pressure to represent hydrostatic mode
    \item Local mass conservation
    \item Stable (not ``stabilized'') velocity space
    \item ALE for moving free surface
    \item $3^3$ point Gauss quadrature
    \item Weakness: not uniformly stable wrt. aspect ratio
    \end{itemize}
  \item Material Point Method
    \begin{itemize}
    \item Lagrangian marker particles
    \item Nonlinearities evaluated at markers
    \item Local $L^2$ projection of coefficients in $Q_1$
    \end{itemize}
  \end{itemize}
\end{frame}

\begin{frame}{Solution procedure}
  \begin{itemize}
  \item Newton linearization
    \begin{itemize}
    \item Picard convergence very slow/stagnates
    \item Implicit or explicit (``drunken seaman'' time step restriction) geometry
    \end{itemize}
  \item Stokes sub-problem
    \begin{equation*}
      \underbrace{\begin{bmatrix} J_{uu} & J_{up} \\ J_{pu} & 0 \end{bmatrix}}_{J} \underbrace{\begin{bmatrix} \bm u \\ p \end{bmatrix}}_{\delta x} = \underbrace{\begin{bmatrix} \bm f_u \\ f_p \end{bmatrix}}_F
    \end{equation*}
  \item Block preconditioning
    \begin{equation*}
      P = \begin{bmatrix} J_{uu} & 0 \\ J_{pu} & S \end{bmatrix}
    \end{equation*}
    \begin{itemize}
    \item $S = -J_{pu} J_{uu}^{-1} J_{up}$ is dense --- not formed
      \begin{itemize}
      \item Approximate by mass matrix weighted by inverse viscosity
        \begin{equation*}
          q^T \tilde S p \sim \int_\Omega \eta^{-1} q p
        \end{equation*}
        Elman, Silvester, Wathen, and others; Grinevich \& Olshanskii (2009); Burstedde (2009); Geenen et al (2009)
      \end{itemize}
    \item \alert{Bottleneck: precondition $J_{uu}$ using multigrid}
    \end{itemize}
  \end{itemize}
\end{frame}

\begin{frame}{Multigrid for viscous block}
  \begin{columns}
    \begin{column}{0.5\textwidth}
      \begin{itemize}
    \item Chebyshev/Jacobi smoothing
      \begin{itemize}
      \item vs. Gauss-Seidel, cf. Adams et al (2003)
      \item Matrix-free implementation
      \item Incomplete factorization viable for anisotropy
      \end{itemize}
    \item Geometric coarsening/rediscretization on finest levels
      \begin{itemize}
      \item Low memory, low setup cost
      \item Switch to Galerkin and AMG on coarser levels
      \end{itemize}
    \end{itemize}
  \end{column}
  \begin{column}{0.5\textwidth}
    \includegraphics[width=\textwidth]{figures/davemay/MayHybridCoarsening.png} \\
  \end{column}
  \end{columns}
  \begin{itemize}
  \item Assembled matrices are memory hogs
    \begin{itemize}
    \item $64\times 64\times 64$ $Q_2$ is $> 12$ GB assembled
    \end{itemize}
  \item Matrix-free operator application
    \begin{itemize}
    \item FE action of $-\nabla\cdot (\kappa \nabla \mathbf u)$
      \begin{gather}\label{eq:mf-scalar}
        A \mathbf u = \sum_{e \, \in \, N_\text{el}}  \mathcal E_e^T \mathcal D_{\mathbf x}^T \Lambda(\omega \kappa) \mathcal D_{\mathbf x} \mathcal E_e \mathbf u,
      \end{gather}
    \end{itemize}
  \end{itemize}
\end{frame}

\begin{frame}{Hardware Arithmetic Intensity}
  \begin{tabular}{lc}
    \toprule
    Operation                         & Arithmetic Intensity (flops/B) \\
    \midrule
    Sparse matrix-vector product      & 1/6                  \\
    Dense matrix-vector product       & 1/4                  \\
    Unassembled matrix-vector product, residual & $\gtrsim 8$          \\
    %High-order residual evaluation    & $\gtrsim 5$                \\
    \bottomrule
  \end{tabular}

  \bigskip

  \begin{tabular}{lrrr}
    \toprule
    Processor & \textsc{stream} Triad (GB/s) & Peak (GF/s) & Balance (F/B) \\
    \midrule
    E5-2680 8-core      & 38   & 173  & 4.5 \\ % http://www.cs.virginia.edu/stream/stream_mail/2013/0001.html
    E5-2695v2 12-core & 45 & 230 & 5.2 \\ % Edison 460 GF and 89 GB/s Triad
    E5-2699v3 18-core & 60 & 660 & 11 \\
    % Magny Cours 16-core & 49   & 281  & 5.7 \\
    Blue Gene/Q node    & 29.3   & 205  & 7 \\ % Morozov's best Triad (https://hpgmg.org/lists/archives/hpgmg-forum/2014-June/000071.html)
    % Tesla M2090         & 120  & 665  & 5.5 \\
    Kepler K20Xm        & 160 & 1310 & 8.2 \\ % http://www.elekslabs.com/2012/11/nvidia-tesla-k20-benchmark-facts.html
    Xeon Phi SE10P      & 161 & 1060 & 6.6 \\ % http://www.cs.virginia.edu/stream/stream_mail/2013/0002.html
    % http://blogs.utexas.edu/jdm4372/tag/xeon-phi/
    \midrule
    KNL (DRAM) & 100 & 3000 & 30 \\
    KNL (MCDRAM) & 500 & 3000 & 6 \\
    \bottomrule
  \end{tabular}
\end{frame}


\begin{frame}{$Q_2$ tensor product optimization}
  \begin{itemize}
  \item Reference gradient $\mathcal D_{\mathbf \xi} = [ \hat D \otimes \hat B \otimes \hat B, \hat B \otimes \hat D \otimes \hat B, \hat B \otimes \hat B \otimes \hat D]$
  \item $\nabla_{\mathbf\xi} \mathbf x = (\mathcal D_{\mathbf\xi}\otimes I_3) (\mathcal E_e\otimes I_3) \mathbf x$, invert $3\times 3$ at quad. points: $\nabla_{\mathbf x} \mathbf \xi$
  \item Viscous operator
    \begin{equation*}
      A \mathbf u = \sum_{e \, \in \, N_\text{el}} \mathcal E_e^T \mathcal D_{\mathbf \xi}^T \Lambda\Big((\nabla_{\mathbf x}\mathbf\xi)^T (\omega \eta) (\nabla_{\mathbf x}\mathbf\xi) \Big) \mathcal D_{\mathbf \xi} \mathcal E_e \mathbf u .
    \end{equation*}
  \item Pack 4 elements at a time in vector-friendly ordering
  \item Intrinsics, $30\%$ of peak AVX (SNB) and FMA (Haswell)
  \item Similar structure in HPGMG-FE (BoF Wed 12:15-13:15 room 294)
  \item 8 nodes (192 cores) of Edison (NERSC XC-30, E5-2695v2):
  \end{itemize}
  \vspace{-1em}
  \begin{center}
    \begin{tabular}{lrrrrrrr}
      \toprule
      Operator & Flops & \multicolumn{2}{c}{Pessimal Cache} & \multicolumn{2}{c}{Perfect Cache} & Time & GF/s \\
               & & Bytes & F/B & Bytes & F/B & (ms) & \\
      \midrule
      Assembled & 9216 & --- & --- &  37248 & 0.247 & 42 & 113 \\
      Matrix-free & 53622 & 2376 & 22.5 & 1008 & 53 & 22 & 651 \\
      Tensor & 15228 & 2376 & 6.4 & 1008 & 15 & \textbf {4.2} & 1072 \\
      Tensor C & 14214 & 5832 & 2.4 & 4920 & 2.9 & --- & --- \\
      \bottomrule
    \end{tabular}
  \end{center}
\end{frame}

\begin{frame}{Sedimentation}
  \begin{columns}
    \begin{column}{0.6\textwidth}
      \begin{itemize}
      \item Randomly-located, dense high-viscosity spheres
      \item Free surface (it matters)
      \item Complicated non-local flow
      \item $n$-sinker much harder than 1-sinker
        \begin{itemize}
        \item not rescued by Krylov
        \end{itemize}
      \item Hierarchy: (AMG), Galerkin, Assembled (Asmb), Matrix-free (MF)
      \end{itemize}
    \end{column}
    \begin{column}{0.4\textwidth}
      \includegraphics[width=\textwidth]{figures/davemay/inclusionsNc75_tubes_spheres.pdf}
    \end{column}
  \end{columns}
  \begin{tabular}{l r r rr rrr}
    % -----------------------
    \toprule
    Grid  &Cores  &Its.  &\multicolumn{2}{l}{Coarse solve}   &\multicolumn{3}{l}{Stokes solve (s)}  \\
          &            &      &Setup (s) &Apply (s)                      &Asmb   &MF    &Tens   \\
          % -----------------------
    \midrule
    $64^3$	&192 &112	&0.5  &1.2		&40.1 &26.9 &14.8 \\
    $96^3$	&192 &110	&1.3  &4.0		&- &90.6 &50.6 \\
    $96^3$	&1536 &95	&1.5  &1.1		&- &12.1 &8.2 \\
    $192^3$	&1536 &141	&2.7  &5.8		&- &120.0 &45.1 \\
    $192^3$	&12288 &170	&4.1  &6.1		&- &30.1 &17.1 \\
    % -----------------------
    \bottomrule
  \end{tabular}
\end{frame}

\begin{frame}{Robustness of block preconditioners}
  \vspace{-1ex}
  \begin{center}
    \includegraphics[width=0.75\textwidth]{figures/davemay/conv-contrast-v.png}
  \end{center}
  \vspace{-1em}
  \begin{itemize}
  \item Non-normality, cf. Schur Complement Reduction
    \begin{equation*}
    \begin{bmatrix} J_{uu} & 0 \\ J_{pu} & S \end{bmatrix}^{-1} \begin{bmatrix} J_{uu} & J_{up} \\ J_{pu} & 0 \end{bmatrix} =
    \begin{bmatrix} I & J_{uu}^{-1} J_{up} \\ 0 & I \end{bmatrix}
  \end{equation*}
  \item Ultimately limited by machine precision
  \end{itemize}
\end{frame}

\begin{frame}{Efficiency: per timestep}
  \begin{itemize}
  \item Global FoM: Elements/second
  \item Efficiency FoM: Elements/core/second
  \end{itemize}
    \begin{tabular}{lr r cr crrr}
    \toprule
    SpMV    &Grid  &Cores    &&{MG res}  &&\multicolumn{3}{l}{{Stokes solve}}      \\
    type         &($E$)    &($C$)        &&TF/s       & &$E$/$C$/s    &GF/$C$/s     &GF/s   \\
    \midrule
    Asmb    &$64^3$  &192     &&0.1       &&46 &0.9 &173          \\
    MF        &$64^3$  &192    &&0.7        &&69 &2.6 &502           \\
    Tens      &$64^3$  &192    &&1.1    &&{\bf 128} &2.4 &464          \\
    \midrule
    MF         &$96^3$   &192     &&0.8      &&58 &2.6 &499           \\
    Tens       &$96^3$  &192     &&1.0     &&{\bf 103} &2.6 &447  \\
    \midrule
    MF           &$96^3$   &1536      &&5.0    &&47 &2.2 &3198    \\
    Tens         &$96^3$   &1536     &&6.6   &&{\bf 72} &2.2 &2378   \\
    \midrule
    MF         &$192^3$   &1536    &&5.3   &&46 &2.5 &3839         \\
    Tens       &$192^3$  &1536     &&7.8   &&{\bf 79} &2.2 &3303   \\
    \midrule
    MF           &$192^3$  &12288  &&36.1   &&19  &1.5 &18499  \\
    Tens        &$192^3$  &12288  &&35.3   &&{\bf 26} &1.1 &12891  \\
    \bottomrule
  \end{tabular}
\end{frame}

\begin{frame}{Outlook}
  \begin{itemize}
  \item pTatin3d: built on PETSc, upcoming public release
  \item Keep science objectives firmly in the drivers' seat
  \item Do not compromise discretization quality
    \begin{itemize}
    \item ``Refine the grid to remove artifacts'' is mathematician code for ``it doesn't work'' or ``I don't know what I'm doing''
    \end{itemize}
  \item Optimize convergence first, then implementation efficiency
  \item Users want performance versatility
    \begin{itemize}
    \item fast solution on supercomputer
    \item high-resolution on small workstation/cluster
    \end{itemize}
  \item Robustness is always a challenge
  \item Tensor product matrix-free a big win for $Q_2$ elements
    \begin{itemize}
    \item Assembled matrices useful to explore algorithms/robustness
    \end{itemize}
  \item Blue Gene/Q performance limited by cache \& integer unit
  \item Thanks to NERSC ``Edison''; Philippe de Clarens and Caroline Baldassari at TOTAL ``Rostand''
  \end{itemize}
\end{frame}


\end{document}
