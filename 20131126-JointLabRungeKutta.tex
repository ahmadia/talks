% \documentclass[handout]{beamer}
\documentclass{beamer}

\mode<presentation>
{
  \usetheme{ANLBlue}
  % \usefonttheme[onlymath]{serif}
  % \usetheme{Singapore}
  % \usetheme{Warsaw}
  % \usetheme{Malmoe}
  % \useinnertheme{circles}
  % \useoutertheme{infolines}
  % \useinnertheme{rounded}

  \setbeamercovered{transparent=20}
}

\usepackage[english]{babel}
\usepackage[latin1]{inputenc}
\usepackage{alltt,listings,multirow,ulem,siunitx}
\usepackage[absolute,overlay]{textpos}
\TPGrid{1}{1}
\usepackage{pdfpages}
\usepackage{ulem}
\usepackage{multimedia}
\usepackage{multicol}
\newcommand\hmmax{0}
\newcommand\bmmax{0}
\usepackage{bm}
\usepackage{comment}
\usepackage{subcaption}

% font definitions, try \usepackage{ae} instead of the following
% three lines if you don't like this look
\usepackage{mathptmx}
\usepackage[scaled=.90]{helvet}
% \usepackage{courier}
\usepackage[T1]{fontenc}
\usepackage{tikz}
\usetikzlibrary{decorations.pathreplacing}
\usetikzlibrary{shadows,arrows,shapes.misc,shapes.arrows,shapes.multipart,arrows,decorations.pathmorphing,backgrounds,positioning,fit,petri,calc,shadows,chains,matrix}

\newcommand\vvec{\bm v}
\newcommand\bvec{\bm b}
\newcommand\bxk{\bvec_0 \times \kappa_0 \cdot \nabla}
\newcommand\delp{\nabla_\perp}

% \usepackage{pgfpages}
% \pgfpagesuselayout{4 on 1}[a4paper,landscape,border shrink=5mm]

\usepackage{JedMacros}

\newcommand{\timeR}{t_{\mathrm{R}}}
\newcommand{\timeW}{t_{\mathrm{W}}}
\newcommand{\mglevel}{\ensuremath{\ell}}
\newcommand{\mglevelcp}{\ensuremath{\mglevel_{\mathrm{cp}}}}
\newcommand{\mglevelcoarse}{\ensuremath{\mglevel_{\mathrm{coarse}}}}
\newcommand{\mglevelfine}{\ensuremath{\mglevel_{\mathrm{fine}}}}

%solution and residual
\newcommand{\vx}{\ensuremath{x}}
\newcommand{\vc}{\ensuremath{\hat{x}}}
\newcommand{\vr}{\ensuremath{r}}
\newcommand{\vb}{\ensuremath{b}}

%operators
\newcommand{\vA}{\ensuremath{A}}
\newcommand{\vP}{\ensuremath{I_H^h}}
\newcommand{\vS}{\ensuremath{S}}
\newcommand{\vR}{\ensuremath{I_h^H}}
\newcommand{\vI}{\ensuremath{\hat I_h^H}}
\newcommand{\vV}{\ensuremath{\mathbf{V}}}
\newcommand{\vF}{\ensuremath{F}}
\newcommand{\vtau}{\ensuremath{\mathbf{\tau}}}


\title{Fast solvers for implicit Runge-Kutta systems}
\author{{\bf Jed Brown} \texttt{jedbrown@mcs.anl.gov} \\
  Debojyoti Ghosh \texttt{ghosh@mcs.anl.gov}
}
\begin{comment}
  As simulation packages mature, they tend to grow ad-hoc interfaces to
  support diverse types of analysis and coupling to other packages.
  These interfaces often involve manufactured impediments such as
  build-time configuration, use of the file system to exchange data, and
  excessive synchronization, which severely impede the algorithms that
  can be used for analysis or coupling.  Meanwhile, hardware is driving
  algorithmic innovation in exposing more parallelism in analysis and
  coupling methods.  We present several such algorithmic trends,
  including recent work within the solvers package PETSc, and
  demonstrate how exposing library-like interfaces is the most
  maintainable and high-performance way for applications to take
  advantage of such methods.
\end{comment}

% - Use the \inst command only if there are several affiliations.
% - Keep it simple, no one is interested in your street address.
\institute
{
  Mathematics and Computer Science Division \\ Argonne National Laboratory
}

\date{JointLab, UIUC, 2013-11-26}

% This is only inserted into the PDF information catalog. Can be left
% out.
\subject{Talks}


% If you have a file called "university-logo-filename.xxx", where xxx
% is a graphic format that can be processed by latex or pdflatex,
% resp., then you can add a logo as follows:

% \pgfdeclareimage[height=0.5cm]{university-logo}{university-logo-filename}
% \logo{\pgfuseimage{university-logo}}



% Delete this, if you do not want the table of contents to pop up at
% the beginning of each subsection:
% \AtBeginSubsection[]
% {
% \begin{frame}<beamer>
%   \frametitle{Outline}
%   \tableofcontents[currentsection,currentsubsection]
% \end{frame}
% }

\AtBeginSection[]
{
  \begin{frame}<beamer>
    \frametitle{Outline}
    \tableofcontents[currentsection]
  \end{frame}
}

% If you wish to uncover everything in a step-wise fashion, uncomment
% the following command:

% \beamerdefaultoverlayspecification{<+->}

\begin{document}
\lstset{language=C}
\normalem

\begin{frame}
  \titlepage
\end{frame}

\section{The memory bandwidth problem}
\begin{frame}{Hardware Arithmetic Intensity}
  \begin{tabular}{lc}
    \toprule
    Operation                         & Arithmetic Intensity (flops/B) \\
    \midrule
    Sparse matrix-vector product      & 1/6                  \\
    Dense matrix-vector product       & 1/4                  \\
    Unassembled matrix-vector product, residual & $\gtrsim 8$          \\
    %High-order residual evaluation    & $\gtrsim 5$                \\
    \bottomrule
  \end{tabular}

  \bigskip

  \begin{tabular}{lrrr}
    \toprule
    Processor & \textsc{stream} Triad (GB/s) & Peak (GF/s) & Balance (F/B) \\
    \midrule
    E5-2680 8-core      & 38   & 173  & 4.5 \\ % http://www.cs.virginia.edu/stream/stream_mail/2013/0001.html
    E5-2695v2 12-core & 45 & 230 & 5.2 \\ % Edison 460 GF and 89 GB/s Triad
    E5-2699v3 18-core & 60 & 660 & 11 \\
    % Magny Cours 16-core & 49   & 281  & 5.7 \\
    Blue Gene/Q node    & 29.3   & 205  & 7 \\ % Morozov's best Triad (https://hpgmg.org/lists/archives/hpgmg-forum/2014-June/000071.html)
    % Tesla M2090         & 120  & 665  & 5.5 \\
    Kepler K20Xm        & 160 & 1310 & 8.2 \\ % http://www.elekslabs.com/2012/11/nvidia-tesla-k20-benchmark-facts.html
    Xeon Phi SE10P      & 161 & 1060 & 6.6 \\ % http://www.cs.virginia.edu/stream/stream_mail/2013/0002.html
    % http://blogs.utexas.edu/jdm4372/tag/xeon-phi/
    \midrule
    KNL (DRAM) & 100 & 3000 & 30 \\
    KNL (MCDRAM) & 500 & 3000 & 6 \\
    \bottomrule
  \end{tabular}
\end{frame}

\begin{frame}{Optimizing Sparse Mat-Vec}
  \begin{itemize}
  \item Order unknowns so vector reuses cache (Cuthill-McKee)
    \begin{itemize}
    \item Optimal: $\frac{(2 \text{ flops})(\text{bandwidth})}{\texttt{sizeof(Scalar)} + \texttt{sizeof(Int)}}$
    \item Usually improves strength of ILU and SOR
    \end{itemize}
  \item Coalesce indices for adjacent rows (Inodes)
    \begin{itemize}
    \item Optimal: $\frac{(2 \text{ flops})(\text{bandwidth})}{\texttt{sizeof(Scalar)} + \texttt{sizeof(Int)}/i}$
    \item Can do block SOR (much stronger than scalar SOR)
    \item Default in PETSc, turn off with \code{-mat\_no\_inode}
    \item Requires ordering unknowns so that fields are interlaced, this
      is (much) better for memory use anyway
    \end{itemize}
  \item Use explicit blocking, hold one index per block (BAIJ format)
    \begin{itemize}
    \item Optimal: $\frac{(2 \text{ flops})(\text{bandwidth})}{\texttt{sizeof(Scalar)} + \texttt{sizeof(Int)}/b^2}$
    \item Block SOR and factorization
    \item Symbolic factorization works with blocks (much cheaper)
    \item Very regular memory access, unrolled dense kernels
    \item Faster insertion: \code{MatSetValuesBlocked()}
    \end{itemize}
  \end{itemize}
\end{frame}

\begin{frame}{This is a dead end}
  \begin{itemize}
  \item Arithmetic intensity $< 1/4$
  \item Idea: multiple right hand sides
    \begin{equation*}
      \frac{(2 k \text{ flops})(\text{bandwidth})}{\texttt{sizeof(Scalar)} + \texttt{sizeof(Int)}}, \quad k \ll \text{avg. nz/row}
    \end{equation*}
  \item Problem: popular algorithms have nested data dependencies
    \begin{itemize}
    \item Time step \\
      \qquad Nonlinear solve \\
      \qquad \qquad Krylov solve \\
      \qquad \qquad \qquad Preconditioner/sparse matrix
    \end{itemize}
  \item Cannot parallelize/vectorize these nested loops
  \end{itemize}
\end{frame}

\begin{frame}{Attempt: $s$-step methods in 3D}
  \includegraphics[width=1.1\textwidth]{figures/SStepEfficiency.pdf}
  \begin{itemize}
  \item Amortizing message latency is most important for strong-scaling
  \item $s$-step methods have high overhead for small subdomains
  \item Limited choice of preconditioners (none optimal)
  \end{itemize}
\end{frame}

\begin{frame}{Attempt: space-time methods (multilevel SDC/Parareal)}
  \includegraphics[width=0.9\textwidth]{figures/EmmettMinionPFASSTCost.png}
  \begin{itemize}
  \item PFASST algorithm (Emmett and Minion, 2013)
  \item Zero-latency messages (cf. performance model of $s$-step)
  \item Spectral Deferred Correction: iterative, converges to IRK (Gauss, Radau, \ldots)
  \item Stiff problems use implicit basic integrator (synchronizing on spatial communicator)
  \end{itemize}
\end{frame}

\begin{frame}{Problems with SDC and time-parallel}
  \includegraphics[width=\textwidth]{figures/TS/SDCScalingEmmett.png} \\
  c/o Matthew Emmett, parallel compared to sequential SDC
  \begin{itemize}
  \item Number of iterations is not uniform, efficiency starts low
  \item Arithmetic intensity unchanged
  \item Parabolic space-time (Greenwald and Brandt/Horton and Vandewalle)
  \end{itemize}
\end{frame}

\section{Implicit Runge-Kutta}
\begin{frame}{Runge-Kutta methods}
  \begin{gather*}
    \dot u = F(u) \\
    \underbrace{
    \begin{pmatrix}
      y_1 \\
      \vdots \\
      y_s
    \end{pmatrix}}_Y =
    u^{n} + h
    \underbrace{
    \begin{bmatrix}
      a_{11} & \dotsb & a_{1s} \\
      \vdots & \ddots & \vdots \\
      a_{s1} & \dotsb & a_{ss}
    \end{bmatrix}}_A
    F
    \begin{pmatrix}
      y_1 \\
      \vdots \\
      y_s
    \end{pmatrix} \\
    u^{n+1} = b^T Y
  \end{gather*}
  \begin{itemize}
  \item General framework for one-step methods
  \item Diagonally implicit: $A$ lower triangular, stage order $\le 2$
  \item Singly diagonally implicit: all $A_{ii}$ equal, reuse solver setup, stage order $\le 1$
  \item If $A$ is a general full matrix, all stages are coupled, ``implicit RK''
  \end{itemize}
\end{frame}

\begin{frame}{Implicit Runge-Kutta}
  \begin{center}
    \begin{tabular}{>{$}c<{$} | >{$}c<{$} >{$}c<{$} >{$}c<{$}}
      \frac 1 2 - \frac{\sqrt{15}}{10} & \frac{5}{36} & \frac 2 9 - \frac{\sqrt{15}}{15} & \frac{5}{36} - \frac{\sqrt{15}}{30} \\
      \frac 1 2 & \frac{5}{36} + \frac{\sqrt{15}}{24} & \frac 2 9 & \frac{5}{36} - \frac{\sqrt{15}}{24} \\
      \frac 1 2 - \frac{\sqrt{15}}{10} & \frac{5}{36} + \frac{\sqrt{15}}{30} & \frac 2 9 + \frac{\sqrt{15}}{15} & \frac{5}{36} \\[4pt]
      \hline
      \vspace{4pt}
      & \frac{5}{18} & \frac 4 9 & \frac{5}{18}
    \end{tabular}
  \end{center}
  \begin{itemize}
  \item Implicit Runge-Kutta methods have excellent accuracy and stability properties
  \item Gauss methods with $s$ stages
    \begin{itemize}
    \item order $2s$, $(s,s)$ Pad\'e approximation to the exponential
    \item $A$-stable, symplectic
    \end{itemize}
  \item Radau (IIA) methods with $s$ stages
    \begin{itemize}
    \item order $2s-1$, $A$-stable, $L$-stable
    \end{itemize}
  \item Lobatto (IIIC) methods with $s$ stages
    \begin{itemize}
    \item order $2s-2$, $A$-stable, $L$-stable, self-adjoint
    \end{itemize}
  \item Stage order $s$ or $s+1$    
  \end{itemize}
\end{frame}

\begin{frame}{Method of Butcher (1976) and Bickart (1977)}
  \begin{itemize}
  \item Newton linearize Runge-Kutta system
    \begin{equation*}
      Y = u^{n} + h A F(Y)
    \end{equation*}
  \item Solve linear system with tensor product operator
    \begin{equation*}
      S \otimes I_n + I_s \otimes J
    \end{equation*}
    where $S = (hA)^{-1}$ is $s\times s$ dense, $J = -\partial F(u)/\partial u$ sparse
  \item SDC (2000) is Gauss-Seidel with low-order corrector
  \item Butcher/Bickart method: diagonalize $S = X \Lambda X^{-1}$
    \begin{itemize}
    \item $\Lambda \otimes I_n + I_s \otimes J$
    \end{itemize}
  \item $s$ decoupled solves
  \item Problem: $X$ is exponentially ill-conditioned wrt. $s$
  \end{itemize}
\end{frame}

\section{Tensor product algebra}
\begin{frame}{MatTAIJ: ``sparse'' tensor product matrices}
  \begin{gather*}
    G = I_n \otimes S + J \otimes T
  \end{gather*}
  \begin{itemize}
  \item More general than multiple RHS (multivectors)
  \item Compare to multiple right hand sides in row-major
  \item Runge-Kutta systems have $T = I_s$ (permuted from Butcher method)
  \item Stream $J$ through cache once, same efficiency as multiple RHS
  \end{itemize}
\end{frame}

\begin{frame}
  128 nodes, 16 procs/node, \emph{small} diffusion problem
  \begin{tabular}{lrrr}
    \toprule
    Method & order & nsteps & time \\
    \midrule
    Gauss 4 & 8 & 10 & 3.4345e-01 \\
    Gauss 2 & 4 & 20 & 7.6320e-01 \\
    Gauss 1 & 2 & 40 & 1.1052e+00 \\
    \bottomrule
  \end{tabular}
\end{frame}

\begin{frame}{Calibration and accuracy}
  \begin{itemize}
  \item Splitting errors plague multi-physics simulation
  \item Verlet (leapfrog) integration is popular: symplectic and \textbf{cheap}
    \begin{itemize}
    \item Stability problems: damping and even/odd decoupling
    \end{itemize}
  \item Models calibrated to compensate
    \begin{itemize}
    \item Force parametrizations in molecular dynamics
    \item Atmospheric column physics
    \end{itemize}
  \end{itemize}
\end{frame}

\begin{frame}
  \includegraphics[width=\textwidth]{figures/TS/CaldwellTimeStepConvergence.png} \\
  %
  c/o Peter Caldwell (LLNL)
  \begin{itemize}
  \item Models calibrated for ``efficient'' time step
  \item Not longer solving the PDEs we write down
  \item Many FTE-years to recalibrate when discretization changes
  \item Calibration eats up a big chunk of the IPCC policy timeline
  \end{itemize}
\end{frame}

\begin{frame}{Implicit Runge-Kutta for advection}
  \begin{table}
    \centering
    \caption{Total number of iterations (communications or accesses of $J$) to solve linear advection to $t=1$ on a $1024$-point grid using point-block Jacobi preconditioning of implicit Runge-Kutta matrix.
      The relative algebraic solver tolerance is $10^{-8}$.}\label{tab:irk-advection}
    \begin{tabular}{lrrr}
      \toprule
      Family & Stages & Order & Iterations \\
      \midrule
      Crank-Nicolson/Gauss & 1 & 2 & 3627 \\
      Gauss & 2 & 4 & 2560 \\
      Gauss & 4 & 8 & 1735 \\
      Gauss & 8 & 16 & 1442 \\
      \bottomrule
    \end{tabular}
  \end{table}
  \begin{itemize}
  \item Naive centered-difference discretization
  \end{itemize}
\end{frame}

% \begin{frame}{IMEX time integration in PETSc}
  \begin{itemize}
  \item Additive Runge-Kutta IMEX methods
    \begin{gather*}
      G(t,x,\dot x) = F(t,x) \\
      J_\alpha = \alpha G_{\dot x} + G_x
    \end{gather*}
    \vspace{-1em}
    \begin{itemize}
    \item User provides:
      \begin{itemize}
      \item \texttt{FormRHSFunction(ts,$t$,$x$,$F$,void *ctx);}
      \item \texttt{FormIFunction(ts,$t$,$x$,$\dot x$,$G$,void *ctx);}
      \item \texttt{FormIJacobian(ts,$t$,$x$,$\dot x$,$\alpha$,$J$,$J_{p}$,mstr,void *ctx);}
      \end{itemize}
    \item Can have $L$-stable DIRK for stiff part $G$, SSP explicit part, etc.
    \item Orders 2 through 5, embedded error estimates
    \item Dense output, hot starts for Newton
    \item More accurate methods if $G$ is linear, also Rosenbrock-W
    \item Can use preconditioner from classical ``semi-implicit'' methods
    \item FAS nonlinear solves supported
    \item Extensible adaptive controllers, can change order within a family
    \item Easy to register new methods: \code{TSARKIMEXRegister()}
    \end{itemize}
  \item Single step interface so user can have own time loop
  \item Same interface for Extrapolation IMEX, LMS IMEX (in development)
  \end{itemize}
\end{frame}


% \begin{frame}[fragile]{Time integration method design}
%   \begin{figure}
%     \centering
%     \includegraphics[width=.8\textwidth]{figures/TS/EmilMethodDesignFeatures.png}
%   \end{figure}
%   \begin{itemize}
%   \item Select order, number of stages, required properties
%   \item Optimize properties like SSP coefficient, accuracy, or linear stability
%   \item \cverb|TSARKIMEXRegister("my-method", ...coefficients...)|
%   \item \cverb|-ts_type arkimex -ts_arkimex_type my-method|
%   \end{itemize}
% \end{frame}

% \begin{frame}\LARGE
%   \begin{itemize}
%   \item Maximize science per Watt
%   \item Huge scope remains at problem formulation
%   \item Raise level of abstraction at which a problem is formally specified
%   \item Algorithmic optimality is crucial
%   \end{itemize}
% \end{frame}

\begin{frame}{Trade-offs in time integration}
  \begin{itemize}
  \item Properties
    \begin{itemize}
    \item Nonlinear stability (e.g., positivity preservation)
    \item Stability along imaginary axis
    \item $L$-stability (damping at infinity)
    \item Implicitness and reuse
    \end{itemize}
  \item What is expensive?
    \begin{itemize}
    \item Function evaluation
    \item Operator assembly/preconditioner setup
      \begin{itemize}
      \item How much can be reused for how long?
      \end{itemize}
    \item Implicit solves
      \begin{itemize}
      \item Can we find better solver algorithm?
      \item More effort in setup?
      \end{itemize}
    \end{itemize}
  \item What is ``convergence''?
    \begin{itemize}
    \item Wave propagation: implicitness useless for convergence \emph{in a norm}
    \item Non-norm functionals could be robust
    \end{itemize}
  \end{itemize}
\end{frame}

\begin{frame}{Outlook}
  \begin{itemize}
  \item Next up: Algebraic multigrid for tensor product operators
  \item Technicalities: imaginary rotation in coarse operators (cf. MG for Helmholtz)
  \item Stochastic Galerkin have some structure
  \item Is it possible to design methods with well-conditioned $S = X \Lambda X^{-1}$
  \end{itemize}
\end{frame}

\end{document}
