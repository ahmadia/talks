% \documentclass[handout]{beamer}
\documentclass{beamer}

\mode<presentation>
{
  \usetheme{ANLBlue}
  % \usefonttheme[onlymath]{serif}
  % \usetheme{Singapore}
  % \usetheme{Warsaw}
  % \usetheme{Malmoe}
  % \useinnertheme{circles}
  % \useoutertheme{infolines}
  % \useinnertheme{rounded}

  \setbeamercovered{transparent=20}
}

\usepackage[english]{babel}
\usepackage[latin1]{inputenc}
\usepackage{alltt,listings,multirow,ulem,siunitx}
\usepackage[absolute,overlay]{textpos}
\TPGrid{1}{1}
\usepackage{pdfpages}
\usepackage{ulem}
\usepackage{multimedia}
\usepackage{multicol}
\newcommand\hmmax{0}
\newcommand\bmmax{0}
\usepackage{bm}
\usepackage{comment}

% font definitions, try \usepackage{ae} instead of the following
% three lines if you don't like this look
\usepackage{mathptmx}
\usepackage[scaled=.90]{helvet}
% \usepackage{courier}
\usepackage[T1]{fontenc}
\usepackage{tikz}
\usetikzlibrary{decorations.pathreplacing}
\usetikzlibrary{shadows,arrows,shapes.misc,shapes.arrows,shapes.multipart,arrows,decorations.pathmorphing,backgrounds,positioning,fit,petri,calc,shadows,chains,matrix}

\newcommand\vvec{\bm v}
\newcommand\bvec{\bm b}
\newcommand\bxk{\bvec_0 \times \kappa_0 \cdot \nabla}
\newcommand\delp{\nabla_\perp}

% \usepackage{pgfpages}
% \pgfpagesuselayout{4 on 1}[a4paper,landscape,border shrink=5mm]

\usepackage{JedMacros}

\newcommand{\timeR}{t_{\mathrm{R}}}
\newcommand{\timeW}{t_{\mathrm{W}}}
\newcommand{\mglevel}{\ensuremath{\ell}}
\newcommand{\mglevelcp}{\ensuremath{\mglevel_{\mathrm{cp}}}}
\newcommand{\mglevelcoarse}{\ensuremath{\mglevel_{\mathrm{coarse}}}}
\newcommand{\mglevelfine}{\ensuremath{\mglevel_{\mathrm{fine}}}}

%solution and residual
\newcommand{\vx}{\ensuremath{x}}
\newcommand{\vc}{\ensuremath{\hat{x}}}
\newcommand{\vr}{\ensuremath{r}}
\newcommand{\vb}{\ensuremath{b}}

%operators
\newcommand{\vA}{\ensuremath{A}}
\newcommand{\vP}{\ensuremath{I_H^h}}
\newcommand{\vS}{\ensuremath{S}}
\newcommand{\vR}{\ensuremath{I_h^H}}
\newcommand{\vI}{\ensuremath{\hat I_h^H}}
\newcommand{\vV}{\ensuremath{\mathbf{V}}}
\newcommand{\vF}{\ensuremath{F}}
\newcommand{\vtau}{\ensuremath{\mathbf{\tau}}}


\title{Inverse problems and uncertainty quantification}
\author{{\bf Jed Brown} \\
\texttt{jedbrown@mcs.anl.gov}
}

% - Use the \inst command only if there are several affiliations.
% - Keep it simple, no one is interested in your street address.
\institute
{
  Mathematics and Computer Science Division \\ Argonne National Laboratory
}

\date{Citcom workshop, 2013-09-16}

% This is only inserted into the PDF information catalog. Can be left
% out.
\subject{Talks}


% If you have a file called "university-logo-filename.xxx", where xxx
% is a graphic format that can be processed by latex or pdflatex,
% resp., then you can add a logo as follows:

% \pgfdeclareimage[height=0.5cm]{university-logo}{university-logo-filename}
% \logo{\pgfuseimage{university-logo}}



% Delete this, if you do not want the table of contents to pop up at
% the beginning of each subsection:
% \AtBeginSubsection[]
% {
% \begin{frame}<beamer>
%   \frametitle{Outline}
%   \tableofcontents[currentsection,currentsubsection]
% \end{frame}
% }

\AtBeginSection[]
{
  \begin{frame}<beamer>
    \frametitle{Outline}
    \tableofcontents[currentsection]
  \end{frame}
}

% If you wish to uncover everything in a step-wise fashion, uncomment
% the following command:

% \beamerdefaultoverlayspecification{<+->}

\begin{document}
\lstset{language=C}
\normalem

\begin{frame}
  \titlepage
\end{frame}

\begin{frame}{Forward models}
  \begin{block}{Deterministic forward modeling with transient PDE}
    \begin{equation*}
      u(t,x; p)
    \end{equation*}
    known parameters $p$: initial/boundary data, coefficients, material models, \ldots
  \end{block}
  \begin{block}{Forward model with uncertainty (posterior)}
    \begin{equation*}
      u(t,x,\mathfrak z; p)
    \end{equation*}
    noise $\mathfrak z$ is stochastic representation of incompletely-modeled processes
  \end{block}
  \begin{block}{Parameter optimization with uncertainty}
    \begin{equation*}
      \min_p \int_{\mathfrak z} f\big(u(t,x,\mathfrak z,p) \big)
    \end{equation*}
  \end{block}
\end{frame}

\begin{frame}{Inversion and design}
  \begin{block}{Data Assimilation}
    \begin{equation*}
      \hat p(t,x,d,d_0) = \argmin_p \int_{\mathfrak y} \int_{\mathfrak z} \norm{d(u(t,x,\mathfrak z,p) + \mathfrak y) - d_0}^2 + \mathrm{Prior}(p)
    \end{equation*}
    infer $p$ from sparse observations $d_0$ with noise $\mathfrak y$
  \end{block}
  \begin{block}{Optimal experimental design}
    \begin{equation*}
      \hat d = \argmin_d \int_p \norm{\hat p(t,x,d,d_0(p)) - p} + \mathrm{cost}(d)
    \end{equation*}    
    choose ``affordable'' sparse observations $d$ to minimize risk in the data assimilation problem over a region of parameter/model space
  \end{block}
\end{frame}

\begin{frame}{Methods for inversion and UQ}
  \begin{itemize}
  \item Stochastic Galerkin/collocation
    \begin{itemize}
    \item Sparse grids, Smolyak quadrature
    \item Accurate in low-dimensional spaces
    \end{itemize}
  \item Bayesian approach
    \begin{equation*}
      \pi_{\text{post}}(x | y_{\text{obs}}) \sim \pi_{\text{prior}} (x) \pi_{\text{like}}(y_{\text{obs}}| x)
    \end{equation*}
    \begin{itemize}
    \item Gaussian prior and linear model: Gaussian posterior
    \item Maximum a posteriori (MAP) point (scalable)
      \begin{equation*}
        \min_{x} \ \ - \log \pi_{\text{post}}(x | y_{\text{obs}})
      \end{equation*}
      like regularized deterministic inversion
    \item MAP point efficient to compute using adjoints
    \item Markov-chain Monte Carlo: high dimensions, nonlinear, but converges slowly
    \end{itemize}
  \end{itemize}
\end{frame}

\end{document}
