% \documentclass[handout]{beamer}
\documentclass{beamer}

\mode<presentation>
{
  \usetheme{ANLBlue}
  % \usefonttheme[onlymath]{serif}
  % \usetheme{Singapore}
  % \usetheme{Warsaw}
  % \usetheme{Malmoe}
  % \useinnertheme{circles}
  % \useoutertheme{infolines}
  % \useinnertheme{rounded}

  \setbeamercovered{transparent=20}
}

\usepackage[english]{babel}
\usepackage[latin1]{inputenc}
\usepackage{alltt,listings,multirow,ulem,siunitx}
\usepackage[absolute,overlay]{textpos}
\TPGrid{1}{1}
\usepackage{pdfpages}
\usepackage{ulem}
\usepackage{multimedia}
\usepackage{multicol}
\newcommand\hmmax{0}
\newcommand\bmmax{0}
\usepackage{bm}
\usepackage{comment}
\usepackage{subcaption}

% font definitions, try \usepackage{ae} instead of the following
% three lines if you don't like this look
\usepackage{mathptmx}
\usepackage[scaled=.90]{helvet}
% \usepackage{courier}
\usepackage[T1]{fontenc}
\usepackage{tikz}
\usetikzlibrary{decorations.pathreplacing}
\usetikzlibrary{shadows,arrows,shapes.misc,shapes.arrows,shapes.multipart,arrows,decorations.pathmorphing,backgrounds,positioning,fit,petri,calc,shadows,chains,matrix}

\newcommand\vvec{\bm v}
\newcommand\bvec{\bm b}
\newcommand\bxk{\bvec_0 \times \kappa_0 \cdot \nabla}
\newcommand\delp{\nabla_\perp}

% \usepackage{pgfpages}
% \pgfpagesuselayout{4 on 1}[a4paper,landscape,border shrink=5mm]

\usepackage{JedMacros}

\newcommand{\timeR}{t_{\mathrm{R}}}
\newcommand{\timeW}{t_{\mathrm{W}}}
\newcommand{\mglevel}{\ensuremath{\ell}}
\newcommand{\mglevelcp}{\ensuremath{\mglevel_{\mathrm{cp}}}}
\newcommand{\mglevelcoarse}{\ensuremath{\mglevel_{\mathrm{coarse}}}}
\newcommand{\mglevelfine}{\ensuremath{\mglevel_{\mathrm{fine}}}}

%solution and residual
\newcommand{\vx}{\ensuremath{x}}
\newcommand{\vc}{\ensuremath{\hat{x}}}
\newcommand{\vr}{\ensuremath{r}}
\newcommand{\vb}{\ensuremath{b}}

%operators
\newcommand{\vA}{\ensuremath{A}}
\newcommand{\vP}{\ensuremath{I_H^h}}
\newcommand{\vS}{\ensuremath{S}}
\newcommand{\vR}{\ensuremath{I_h^H}}
\newcommand{\vI}{\ensuremath{\hat I_h^H}}
\newcommand{\vV}{\ensuremath{\mathbf{V}}}
\newcommand{\vF}{\ensuremath{F}}
\newcommand{\vtau}{\ensuremath{\mathbf{\tau}}}


\title{HPGMG: High-Performance Geometric Multigrid}
\author{{\bf Jed Brown} \texttt{jedbrown@mcs.anl.gov} (ANL and CU Boulder)
}

% - Use the \inst command only if there are several affiliations.
% - Keep it simple, no one is interested in your street address.
% \institute
% {
%   Mathematics and Computer Science Division \\ Argonne National Laboratory
% }

\date{$[HPC]^3$, KAUST, 2014-11-10 \\[1em]
{\small This talk: \url{http://59A2.org/files/20141110-HPGMG.pdf}}}

% This is only inserted into the PDF information catalog. Can be left
% out.
\subject{Talks}


% If you have a file called "university-logo-filename.xxx", where xxx
% is a graphic format that can be processed by latex or pdflatex,
% resp., then you can add a logo as follows:

% \pgfdeclareimage[height=0.5cm]{university-logo}{university-logo-filename}
% \logo{\pgfuseimage{university-logo}}



% Delete this, if you do not want the table of contents to pop up at
% the beginning of each subsection:
% \AtBeginSubsection[]
% {
% \begin{frame}<beamer>
%   \frametitle{Outline}
%   \tableofcontents[currentsection,currentsubsection]
% \end{frame}
% }

% \AtBeginSection[]
% {
%   \begin{frame}<beamer>
%     \frametitle{Outline}
%     \tableofcontents[currentsection]
%   \end{frame}
% }

% If you wish to uncover everything in a step-wise fashion, uncomment
% the following command:

% \beamerdefaultoverlayspecification{<+->}

\begin{document}
\lstset{language=C}
\normalem

\begin{frame}{HPGMG: a new benchmarking proposal}
  \begin{itemize}
  \item \url{https://hpgmg.org}, hpgmg-forum@hpgmg.org mailing list
  \item SC14 BoF: Wednesday, Nov 19, 12:15pm to 1:15pm
  \item Mark Adams, Sam Williams (finite-volume), myself (finite-element), John Shalf, Brian Van Straalen, Erich Strohmeier, Rich Vuduc
  \item Implementations
    \begin{description}
    \item[Finite Volume] memory bandwidth intensive, simple data dependencies
    \item[Finite Element] compute- and cache-intensive, vectorizes, overlapping writes
    \end{description}
  \item Full multigrid, well-defined, scale-free problem
  \item Goal: necessary and sufficient
    \begin{itemize}
    \item Every feature stressed by benchmark should be necessary for an important application
    \item Good performance on the benchmark should be sufficient for good performance on most applications
    \end{itemize}
  \end{itemize}
\end{frame}

\begin{frame}{Kiviat diagrams}
  \begin{center}
    \includegraphics[width=\textwidth]{figures/hpgmg-kiviat-20140606.png}
  \end{center}
  \begin{itemize}
  \item c/o Ian Karlin and Bert Still (LLNL)
  \end{itemize}
\end{frame}

\begin{frame}{HPGMG distinguishes networks at 1M dofs/core}
  \begin{center}
    \includegraphics[width=0.6\textwidth]{figures/hpgmg-fv-20140515-dof.png}
  \end{center}
  \begin{itemize}
  \item Peregrine and Edison have identical node architecture
  \item Peregrine has 5:1 tapered IB, Edison has Aries dragonfly topology
  \end{itemize}
\end{frame}

\begin{frame}
  \vspace{-1em}
  \begin{center}
    \includegraphics[width=0.9\textwidth]{figures/MG/titan-edison-supermuc-range.png}
  \end{center}
  \vspace{-1em}
  \begin{itemize}
  \item Turn-around time often not negotiable
    \begin{itemize}
    \item policy, manufacturing, forecasting
    \end{itemize}
  \item Users like predictable performance across a range of problem sizes
  \item Transient problems do not weak scale even if each step does
  \end{itemize}
\end{frame}

\begin{frame}{Where we are now: $QR$ factorization with MKL on MIC}
  \begin{figure}
    \centering
    \includegraphics[width=\textwidth]{figures/hardware/MKL-dgeqrf-MIC-201411.png}
  \end{figure}
  \begin{itemize}
  \item Figure compares two CPU sockets (230W TDP) to one MIC (300W TDP plus host)
  \item Performance/Watt only breaks even at largest problem sizes
  \item Haswell-EP doubles performance within same power envelope
  \item $10^4 \times 10^4$ matrix takes 667 GFlops: about 2 seconds
  \item This is an $O(n^{3/2})$ operation on $n$ data
  \item MIC cannot strong scale, no more energy efficient/cost effective
  \item ``hard to program'' versus ``architecture ill-suited for problem''?
  \end{itemize}
\end{frame}

\end{document}
