\documentclass{beamer}

\usetheme{ANLBlue}
\setbeamertemplate{navigation symbols}{}

% LCM: Use bold for frametitle
% But this elimiates subtitles ... How to do bold for frametitle and keep subtitle?
% \setbeamertemplate{frametitle} {\textbf{\insertframetitle}}

% Delete this, if you do not want the table of contents to pop up at
% the beginning of each subsection:  
\AtBeginSection[]
{
  \begin{frame}<beamer>
    \frametitle{Outline}
    \tableofcontents[currentsection,hideothersubsections]
  \end{frame}
}

\AtBeginSubsection[]
{
  \begin{frame}<beamer>
    \frametitle{Outline}
    \tableofcontents[sectionstyle=show/hide,subsectionstyle=show/shaded/hide]
  \end{frame}
}

% \usepackage{tikz}
% \usetikzlibrary{shadows,arrows,shapes.misc,shapes.arrows,arrows,decorations.pathmorphing,backgrounds,positioning,fit,petri,calc,shadows,chains,matrix}

\usepackage{pgfplots}

\usepackage{graphicx}

\usepackage{array}
% \usepackage{ulem} % uses underlining for emphasis
\usepackage{fancyvrb}
% \usepackage[absolute,overlay,showboxes]{textpos}
\usepackage[absolute,overlay]{textpos}
\TPGrid{1}{1}
% Absolute positioning is done with fractions of a slide
% \begin{textblock}{BOXWIDTH}[H0,V0](HLOC,VLOC)
%   content goes here, perhaps \includegraphics
% \end{textblock}
% BOXWIDTH : width of text box
% [H0,V0] : location in text box to use for positioning
%    [0,1] is the southwest corner, [0.5,0.5] is the center, [1,0.5] is the
%    east edge, etc
% (HLOC,VLOC) : location in the frame to place the reference point
%    [1,0](0,0.5) places the upper left corner of the box at the right
%    midpoint of the text box.

% Omit label for subfigure

% Text macros
\newenvironment{changemargin}[2]{%
  \begin{list}{}{%
    \setlength{\topsep}{0pt}%
    \setlength{\leftmargin}{#1}%
    \setlength{\rightmargin}{#2}%
    \setlength{\listparindent}{\parindent}%
    \setlength{\itemindent}{\parindent}%
    \setlength{\parsep}{\parskip}%
  }%
  \item[]}{\end{list}}


% If you wish to uncover everything in a step-wise fashion, uncomment
% the following command: 

%\beamerdefaultoverlayspecification{<+->}     

\begin{document}

\title{\bf Interactive solver design using eigen-analysis}

\begin{frame}[shrink=5]{Interactive solver design using eigen-analysis}
  \textbf{\small Jed Brown, Matt Otten, Barry Smith} \\
  Why isn't my solver converging?
  \begin{equation*}
    T = \underbrace{(1 - M^{-1}A)}_{\text{post-smooth} T_s} \underbrace{(1 - P A_c^{-1}R A)}_{\text{coarse-correction} T_c} \underbrace{(1 - M^{-1}A)}_{\text{pre-smooth} T_s}
  \end{equation*}
  \vspace{-1.5em}
  \begin{columns}
    \begin{column}{0.6\textwidth}
      \begin{itemize}
      \item PETSc plugin uses SLEPc to compute eigenpairs.
      \item MG, DD, and field-split solvers are based on subspace correction: fast when complementary work is done in each subspace.
      \item Idealized smoother $T_s = PR$, idealized coarse correction $T_c = 1 - PR$.
      \item SAWs: Scientific Application Web server, embedded, JS client, RESTful interface (Matt Otten)
      \end{itemize}
    \end{column}
    \begin{column}{0.4\textwidth}
      \begin{figure}
        \centering
        \includegraphics[width=\textwidth]{figures/THI/EigenGAMG/gamg-lambda0.png}
        \caption{$\lambda_0 = 0.0268$. \small Smoothed aggregation w/o rotational modes, hydrostatic ice flow, bumpy, slippery bed.}
      \end{figure}
    \end{column}
  \end{columns}
\end{frame}

\end{document}
